% Options for packages loaded elsewhere
% Options for packages loaded elsewhere
\PassOptionsToPackage{unicode}{hyperref}
\PassOptionsToPackage{hyphens}{url}
\PassOptionsToPackage{dvipsnames,svgnames,x11names}{xcolor}
%
\documentclass[
  11pt,
]{article}
\usepackage{xcolor}
\usepackage[margin=1in]{geometry}
\usepackage{amsmath,amssymb}
\setcounter{secnumdepth}{5}
\usepackage{iftex}
\ifPDFTeX
  \usepackage[T1]{fontenc}
  \usepackage[utf8]{inputenc}
  \usepackage{textcomp} % provide euro and other symbols
\else % if luatex or xetex
  \usepackage{unicode-math} % this also loads fontspec
  \defaultfontfeatures{Scale=MatchLowercase}
  \defaultfontfeatures[\rmfamily]{Ligatures=TeX,Scale=1}
\fi
\usepackage{lmodern}
\ifPDFTeX\else
  % xetex/luatex font selection
\fi
% Use upquote if available, for straight quotes in verbatim environments
\IfFileExists{upquote.sty}{\usepackage{upquote}}{}
\IfFileExists{microtype.sty}{% use microtype if available
  \usepackage[]{microtype}
  \UseMicrotypeSet[protrusion]{basicmath} % disable protrusion for tt fonts
}{}
\makeatletter
\@ifundefined{KOMAClassName}{% if non-KOMA class
  \IfFileExists{parskip.sty}{%
    \usepackage{parskip}
  }{% else
    \setlength{\parindent}{0pt}
    \setlength{\parskip}{6pt plus 2pt minus 1pt}}
}{% if KOMA class
  \KOMAoptions{parskip=half}}
\makeatother
% Make \paragraph and \subparagraph free-standing
\makeatletter
\ifx\paragraph\undefined\else
  \let\oldparagraph\paragraph
  \renewcommand{\paragraph}{
    \@ifstar
      \xxxParagraphStar
      \xxxParagraphNoStar
  }
  \newcommand{\xxxParagraphStar}[1]{\oldparagraph*{#1}\mbox{}}
  \newcommand{\xxxParagraphNoStar}[1]{\oldparagraph{#1}\mbox{}}
\fi
\ifx\subparagraph\undefined\else
  \let\oldsubparagraph\subparagraph
  \renewcommand{\subparagraph}{
    \@ifstar
      \xxxSubParagraphStar
      \xxxSubParagraphNoStar
  }
  \newcommand{\xxxSubParagraphStar}[1]{\oldsubparagraph*{#1}\mbox{}}
  \newcommand{\xxxSubParagraphNoStar}[1]{\oldsubparagraph{#1}\mbox{}}
\fi
\makeatother


\usepackage{longtable,booktabs,array}
\usepackage{calc} % for calculating minipage widths
% Correct order of tables after \paragraph or \subparagraph
\usepackage{etoolbox}
\makeatletter
\patchcmd\longtable{\par}{\if@noskipsec\mbox{}\fi\par}{}{}
\makeatother
% Allow footnotes in longtable head/foot
\IfFileExists{footnotehyper.sty}{\usepackage{footnotehyper}}{\usepackage{footnote}}
\makesavenoteenv{longtable}
\usepackage{graphicx}
\makeatletter
\newsavebox\pandoc@box
\newcommand*\pandocbounded[1]{% scales image to fit in text height/width
  \sbox\pandoc@box{#1}%
  \Gscale@div\@tempa{\textheight}{\dimexpr\ht\pandoc@box+\dp\pandoc@box\relax}%
  \Gscale@div\@tempb{\linewidth}{\wd\pandoc@box}%
  \ifdim\@tempb\p@<\@tempa\p@\let\@tempa\@tempb\fi% select the smaller of both
  \ifdim\@tempa\p@<\p@\scalebox{\@tempa}{\usebox\pandoc@box}%
  \else\usebox{\pandoc@box}%
  \fi%
}
% Set default figure placement to htbp
\def\fps@figure{htbp}
\makeatother





\setlength{\emergencystretch}{3em} % prevent overfull lines

\providecommand{\tightlist}{%
  \setlength{\itemsep}{0pt}\setlength{\parskip}{0pt}}



 


\usepackage{booktabs}
\usepackage{longtable}
\usepackage{array}
\usepackage{multirow}
\usepackage{wrapfig}
\usepackage{float}
\usepackage{colortbl}
\usepackage{pdflscape}
\usepackage{tabu}
\usepackage{threeparttable}
\usepackage{threeparttablex}
\usepackage[normalem]{ulem}
\usepackage{makecell}
\usepackage{xcolor}
\usepackage{booktabs}
\usepackage{dcolumn}
\usepackage{longtable}
\usepackage{adjustbox}
\usepackage{amsmath}
\makeatletter
\@ifpackageloaded{caption}{}{\usepackage{caption}}
\AtBeginDocument{%
\ifdefined\contentsname
  \renewcommand*\contentsname{Table of contents}
\else
  \newcommand\contentsname{Table of contents}
\fi
\ifdefined\listfigurename
  \renewcommand*\listfigurename{List of Figures}
\else
  \newcommand\listfigurename{List of Figures}
\fi
\ifdefined\listtablename
  \renewcommand*\listtablename{List of Tables}
\else
  \newcommand\listtablename{List of Tables}
\fi
\ifdefined\figurename
  \renewcommand*\figurename{Figure}
\else
  \newcommand\figurename{Figure}
\fi
\ifdefined\tablename
  \renewcommand*\tablename{Table}
\else
  \newcommand\tablename{Table}
\fi
}
\@ifpackageloaded{float}{}{\usepackage{float}}
\floatstyle{ruled}
\@ifundefined{c@chapter}{\newfloat{codelisting}{h}{lop}}{\newfloat{codelisting}{h}{lop}[chapter]}
\floatname{codelisting}{Listing}
\newcommand*\listoflistings{\listof{codelisting}{List of Listings}}
\makeatother
\makeatletter
\makeatother
\makeatletter
\@ifpackageloaded{caption}{}{\usepackage{caption}}
\@ifpackageloaded{subcaption}{}{\usepackage{subcaption}}
\makeatother
\usepackage{bookmark}
\IfFileExists{xurl.sty}{\usepackage{xurl}}{} % add URL line breaks if available
\urlstyle{same}
\hypersetup{
  pdftitle={Quarterly Event Study: Eviction Filing Rates Around Ownership Transfers},
  pdfauthor={Philly Evictions Project},
  colorlinks=true,
  linkcolor={blue},
  filecolor={Maroon},
  citecolor={Blue},
  urlcolor={Blue},
  pdfcreator={LaTeX via pandoc}}


\title{Quarterly Event Study: Eviction Filing Rates Around Ownership
Transfers}
\author{Philly Evictions Project}
\date{2026-02-25}
\begin{document}
\maketitle

\renewcommand*\contentsname{Table of contents}
{
\hypersetup{linkcolor=}
\setcounter{tocdepth}{3}
\tableofcontents
}

\section{Motivation}\label{motivation}

The annual event study (\texttt{analyze-transfer-evictions.R}) shows a
suspicious jump in filing rates from \(t=-2\) to \(t=-1\) (the year
before transfer). Monthly analysis of raw filing dates reveals the true
pattern: \textbf{filings decline in the 3--6 months before transfer and
bottom out at the transfer date}, then recover afterward.

The annual binning creates artifacts because \(t=-1\) (the calendar year
before transfer) mixes the elevated pre-decline period with the
declining period, depending on when in the year the transfer occurs.

\textbf{This quarterly event study uses exact transfer dates and
day-level filing dates} to:

\begin{enumerate}
\def\labelenumi{\arabic{enumi}.}
\tightlist
\item
  Correctly identify \emph{when} the filing rate change occurs relative
  to the actual transfer
\item
  Distinguish the pattern by building size (the pre-transfer dip for
  1-unit buildings could be vacancy/non-renewal, while multi-unit
  buildings would indicate an operational pattern)
\item
  Give cleaner identification of the post-transfer ramp-up timing
\end{enumerate}

\section{Data Construction}\label{data-construction}

\subsection{Quarterly eviction counts}\label{quarterly-eviction-counts}

Day-level eviction filing dates (\texttt{d\_filing} from
\texttt{evictions\_clean}) are linked to parcels via the address
crosswalk (\texttt{evict\_address\_xwalk}, restricted to unique parcel
matches). Filings are aggregated to PID \(\times\) quarter cells using
the \texttt{YYYY-Q\#} format.

\subsection{Event panel}\label{event-panel}

Transfer events from \texttt{rtt\_clean} use the exact
\texttt{display\_date} for timing. We:

\begin{enumerate}
\def\labelenumi{\arabic{enumi}.}
\tightlist
\item
  Collapse RTT to PID-quarter level (keeping the highest-consideration
  transfer per quarter)
\item
  Expand to a quarterly event grid: \(q \in \{-12, \ldots, +20\}\) (3
  years before, 5 years after)
\item
  Map each relative quarter to a calendar quarter by adding
  \(q \times 91\) days to the actual transfer date
\item
  Merge quarterly filing counts (zeros where no filings)
\item
  Merge annual \texttt{total\_units} from \texttt{bldg\_panel\_blp} for
  rate computation
\item
  Compute filing rate: \(\text{num\_filings\_q} / \text{total\_units}\)
\end{enumerate}

\textbf{Sample:} The baseline regression uses 5,721,271 PID \(\times\)
quarter observations (full sample, unweighted). The 5+ unit subsample
has 79,423 observations.

\textbf{Pre-COVID restriction:} All filing rate outcomes are restricted
to calendar quarters \(\leq\) 2019Q4 (pre-COVID). The pandemic disrupted
court operations and filing patterns, making post-2019 data unreliable
for measuring landlord behavior.

\textbf{Standard errors} are clustered at the PID level throughout ---
the building is the unit of treatment and where serial correlation in
filing rates lives.

\section{Raw Quarterly Profile}\label{raw-quarterly-profile}

\begin{figure}[H]

{\centering \includegraphics[width=1\linewidth,height=\textheight,keepaspectratio]{../output/figs/rtt_eviction_quarterly_raw_profile.png}

}

\caption{Mean quarterly filing rate around ownership transfer (full
sample). The pre-transfer dip at q=-1 and q=0 is clearly visible.}

\end{figure}%

\begin{figure}[H]

{\centering \includegraphics[width=1\linewidth,height=\textheight,keepaspectratio]{../output/figs/rtt_eviction_quarterly_raw_1v_multi.png}

}

\caption{Mean quarterly filing rate: 1-unit vs multi-unit buildings. The
pre-transfer dip is much sharper for single-unit buildings, consistent
with vacancy during sale.}

\end{figure}%

The quarterly profile reveals a clear pattern that the annual analysis
obscures:

\begin{enumerate}
\def\labelenumi{\arabic{enumi}.}
\tightlist
\item
  \textbf{Baseline period} (\(q = -12\) to \(q = -6\)): Filing rates are
  relatively stable around 0.010 per unit-quarter.
\item
  \textbf{Pre-transfer ramp} (\(q = -5\) to \(q = -2\)): Rates rise from
  0.011 to 0.013. Likely reflects sellers increasing activity or
  properties with rising tenant problems being selected into transfer.
\item
  \textbf{Pre-transfer dip} (\(q = -1\)): Sharp drop to 0.010 --- a 27\%
  decline from the \(q = -2\) peak.
\item
  \textbf{Transfer quarter} (\(q = 0\)): Bottom at 0.006. The new owner
  has just taken possession.
\item
  \textbf{Post-transfer recovery} (\(q = +1\) to \(q = +3\)): Rapid
  recovery from 0.008 to 0.012.
\item
  \textbf{New steady state} (\(q = +4\) onward): Rates stabilize at
  0.013--0.014, approximately 35\% above the pre-transfer baseline.
\end{enumerate}

\textbf{The 1-unit vs multi-unit split} is informative: the pre-transfer
dip is much deeper for single-unit buildings, consistent with vacancy
during the sale process. Multi-unit buildings maintain relatively stable
filing rates through the transition.

\begin{figure}[H]

{\centering \includegraphics[width=1\linewidth,height=\textheight,keepaspectratio]{../output/figs/rtt_eviction_quarterly_by_size_4way.png}

}

\caption{Mean quarterly filing rate by building size (4-way split).}

\end{figure}%

\section{Specification}\label{specification}

The baseline quarterly event study estimates:

\[
\text{FilingRate}_{it}^{(q)} = \sum_{k \neq -4} \beta_k \cdot \mathbf{1}[\text{q\_relative}_{it} = k] + \alpha_i + \gamma_{yq} + \varepsilon_{it}
\]

where \(i\) indexes buildings (PIDs), \(yq\) indexes calendar
year-quarters, and \(k \in \{-12, \ldots, -5, -3, \ldots, +20\}\) with
\(k = -4\) (one year before transfer) as the omitted reference period.
We use \(q = -4\) rather than \(q = -1\) because the pre-transfer dip
begins around \(q = -1\), making it a poor reference point.

\(\alpha_i\) are building fixed effects and \(\gamma_{yq}\) are
year-quarter fixed effects. Standard errors are clustered at the PID
level.

\textbf{Unit-weighted specifications} weight each observation by
\texttt{total\_units}, so a 50-unit building contributes 50\(\times\)
the weight of a single-family home. This answers ``what happens to the
average \emph{rental unit}'' rather than ``what happens to the average
\emph{building}.''

\section{Baseline Regression Results}\label{baseline-regression-results}

\begin{figure}[H]

{\centering \includegraphics[width=1\linewidth,height=\textheight,keepaspectratio]{../output/figs/rtt_eviction_quarterly_baseline.png}

}

\caption{Quarterly event study coefficients (faceted). Eight
specifications: full sample (weighted and unweighted), GEOID\(\times\)yq
FE, 5+ units (weighted and unweighted), known rentals 5+, 2+ units, and
known rentals (all). Reference: q=-4.}

\end{figure}%

\begingroup\fontsize{8}{10}\selectfont

\begin{longtable}[t]{rcccccccc}
\caption{\label{tab:baseline-table}Selected quarterly event study coefficients (SE in parentheses). Reference: q=-4.}\\
\toprule
q & 2plus\_unweighted & 5plus\_unit\_weighted & 5plus\_unweighted & full\_geoid\_x\_yq & full\_unit\_weighted & full\_unweighted & known\_rental\_5plus & known\_rental\_all\\
\midrule
-8 & -0.00048 (0.00058) & -0.00128 (0.00176) & 0.00012 (0.00126) & -0.00220 (0.00037) & -0.00183 (0.00044) & -0.00227 (0.00037) & -0.00129 (0.00289) & -0.01275 (0.00131)\\
-2 & 0.00090 (0.00060) & -0.00064 (0.00150) & -0.00046 (0.00142) & 0.00181 (0.00039) & 0.00124 (0.00041) & 0.00182 (0.00039) & -0.00093 (0.00307) & 0.00403 (0.00144)\\
-1 & -0.00090 (0.00058) & -0.00173 (0.00165) & -0.00055 (0.00155) & -0.00209 (0.00036) & -0.00185 (0.00041) & -0.00209 (0.00036) & -0.00105 (0.00337) & -0.01240 (0.00130)\\
0 & -0.00161 (0.00059) & -0.00304 (0.00163) & -0.00234 (0.00136) & -0.00509 (0.00034) & -0.00412 (0.00041) & -0.00505 (0.00034) & -0.00534 (0.00285) & -0.02364 (0.00121)\\
1 & 0.00013 (0.00061) & 0.00248 (0.00212) & 0.00224 (0.00158) & -0.00340 (0.00036) & -0.00176 (0.00049) & -0.00342 (0.00036) & 0.00308 (0.00307) & -0.02036 (0.00122)\\
\addlinespace
2 & -0.00003 (0.00059) & -0.00194 (0.00167) & 0.00081 (0.00144) & -0.00170 (0.00037) & -0.00148 (0.00043) & -0.00166 (0.00037) & -0.00018 (0.00273) & -0.01655 (0.00123)\\
4 & 0.00196 (0.00061) & 0.00052 (0.00168) & 0.00268 (0.00146) & 0.00192 (0.00040) & 0.00171 (0.00044) & 0.00195 (0.00039) & 0.00256 (0.00273) & -0.00935 (0.00125)\\
8 & 0.00145 (0.00061) & 0.00159 (0.00174) & 0.00219 (0.00164) & 0.00306 (0.00041) & 0.00272 (0.00046) & 0.00319 (0.00042) & 0.00110 (0.00280) & -0.01008 (0.00125)\\
\bottomrule
\end{longtable}
\endgroup{}

Key comparisons:

\begin{itemize}
\tightlist
\item
  \textbf{Unweighted vs unit-weighted}: Unit-weighting upweights large
  buildings. The weighted estimates answer ``what happens to the average
  rental unit after a transfer?'' while unweighted answers ``what
  happens to the average building?''
\item
  \textbf{GEOID \(\times\) yq FE}: Adding block-group by year-quarter
  fixed effects produces nearly identical estimates, confirming the
  pattern is not driven by neighborhood trends.
\item
  \textbf{5+ units}: The most relevant subsample --- these are
  established rental properties where the vacancy artifact is minimal.
\item
  \textbf{2+ units}: Drops single-family homes (which dominate the full
  sample and have large vacancy artifacts) while retaining smaller
  multi-family properties.
\item
  \textbf{Known rentals}: Restricts to properties with observed rental
  activity (license, filing, or InfoUSA match), removing owner-occupied
  buildings from the sample.
\end{itemize}

\section{Building Size Heterogeneity}\label{sec-size}

\begin{figure}[H]

{\centering \includegraphics[width=1\linewidth,height=\textheight,keepaspectratio]{../output/figs/rtt_eviction_quarterly_by_size.png}

}

\caption{Quarterly event study by building size (4-way split, faceted).
Separate regressions with PID + yq FE, clustered at PID.}

\end{figure}%

\begingroup\fontsize{9}{11}\selectfont

\begin{longtable}[t]{rcccc}
\caption{\label{tab:size-table}Selected quarterly coefficients by building size. Reference: q=-4.}\\
\toprule
q & size\_1 unit & size\_10+ units & size\_2-4 units & size\_5-9 units\\
\midrule
-8 & -0.00268 & -0.00115 & -0.00052 & 0.00093\\
-2 & 0.00202 & -0.00018 & 0.00101 & -0.00071\\
-1 & -0.00236 & -0.00289 & -0.00094 & 0.00105\\
0 & -0.00583 & -0.00277 & -0.00156 & -0.00213\\
1 & -0.00422 & 0.00414 & -0.00005 & 0.00075\\
\addlinespace
4 & 0.00194 & 0.00093 & 0.00190 & 0.00406\\
8 & 0.00358 & 0.00116 & 0.00140 & 0.00297\\
\bottomrule
\end{longtable}
\endgroup{}

\textbf{Key finding:} The pre-transfer dip is concentrated in small
buildings. For 10+ unit buildings, filing rates are stable through the
transition, confirming the dip is a vacancy artifact rather than an
operational pattern.

\section{Buyer Type Heterogeneity (5+ Units)}\label{sec-buyer}

Buyer-type regressions are shown for both \textbf{5+ unit} and
\textbf{2+ unit} buildings to avoid the single-family vacancy artifact.

\begin{figure}[H]

{\centering \includegraphics[width=1\linewidth,height=\textheight,keepaspectratio]{../output/figs/rtt_eviction_quarterly_by_buyer.png}

}

\caption{Quarterly event study by buyer type (faceted). Corporate vs
individual splits for 5+ and 2+ unit buildings. PID + yq FE, clustered
at PID.}

\end{figure}%

\begin{longtable}[t]{rcccc}
\caption{\label{tab:buyer-table}Quarterly event study by buyer type, 5+ units (SE in parentheses). Reference: q=-4.}\\
\toprule
q & corporate\_2plus & corporate\_5plus & individual\_2plus & individual\_5plus\\
\midrule
-8 & -0.00016 (0.00102) & 0.00004 (0.00166) & -0.00066 (0.00070) & 0.00036 (0.00166)\\
-2 & -0.00113 (0.00096) & 0.00031 (0.00186) & 0.00214 (0.00077) & -0.00224 (0.00190)\\
-1 & -0.00231 (0.00096) & -0.00051 (0.00200) & -0.00011 (0.00073) & -0.00058 (0.00224)\\
0 & -0.00201 (0.00106) & -0.00247 (0.00172) & -0.00146 (0.00072) & -0.00195 (0.00215)\\
1 & 0.00151 (0.00110) & 0.00309 (0.00201) & -0.00080 (0.00072) & 0.00043 (0.00228)\\
\addlinespace
2 & -0.00005 (0.00101) & 0.00173 (0.00179) & -0.00018 (0.00073) & -0.00152 (0.00228)\\
4 & 0.00171 (0.00107) & 0.00264 (0.00178) & 0.00189 (0.00076) & 0.00264 (0.00219)\\
8 & 0.00353 (0.00118) & 0.00280 (0.00204) & 0.00009 (0.00072) & 0.00038 (0.00245)\\
12 & 0.00224 (0.00124) & 0.00247 (0.00235) & 0.00057 (0.00073) & -0.00133 (0.00253)\\
\bottomrule
\end{longtable}

\section{Acquirer Filing Rate Heterogeneity}\label{sec-acq}

We classify acquirers by their \textbf{pre-acquisition temporal filing
rate} on their other properties. For each transfer of PID \(i\) by owner
\(j\) (transfer year \(Y\)), we compute the mean
\texttt{num\_filings\ /\ total\_units} across all \emph{other} PIDs that
owner \(j\) acquired, using only years strictly before \(Y\). This
avoids both mechanical circularity and any look-ahead bias.

\textbf{Definitions:}

\begin{itemize}
\tightlist
\item
  \textbf{High-filer portfolio}: Pre-acquisition rate \(>\) 0.0479 (the
  median among acquirers with rate \(> 0\)), and the owner acquired at
  least one other property
\item
  \textbf{Low-filer portfolio}: \(0 <\) rate \(\leq\) 0.0479, has other
  properties
\item
  \textbf{Non-filer (has portfolio)}: Rate \(= 0\) but has other
  properties (no pre-acquisition filings observed)
\item
  \textbf{Single-purchase}: No other matched properties (first buy or
  unmatched)
\end{itemize}

\begingroup\fontsize{9}{11}\selectfont

\begin{longtable}[t]{lrrrrrrrr}
\caption{\label{tab:acq-stats-table}Acquirer classification (2+ unit buildings, at q=0). Rate = mean pre-acquisition filing rate on acquirer's OTHER properties.}\\
\toprule
Acquirer type & Transfers & PIDs & Owners & Total units & Mean acq rate & Mean n other & \% transfers & \% units\\
\midrule
Single-purchase & 26,145 & 19,531 & 25,793 & 106,264 & NA & 0.0 & 60.4 & 66.8\\
High-filer portfolio & 4,365 & 3,924 & 2,474 & 16,697 & 0.1216 & 80.1 & 10.1 & 10.5\\
Non-filer (has portfolio) & 8,316 & 7,399 & 6,136 & 22,798 & 0.0000 & 2.8 & 19.2 & 14.3\\
Low-filer portfolio & 4,442 & 4,139 & 1,974 & 13,435 & 0.0258 & 37.5 & 10.3 & 8.4\\
\bottomrule
\end{longtable}
\endgroup{}

\subsection{2+ unit buildings}\label{unit-buildings}

\begin{figure}[H]

{\centering \includegraphics[width=1\linewidth,height=\textheight,keepaspectratio]{../output/figs/rtt_eviction_quarterly_by_acq_filer.png}

}

\caption{Quarterly event study by acquirer pre-acquisition filing rate
(2+ unit buildings). High-filer portfolio acquirers show the largest
post-transfer increase.}

\end{figure}%

\begingroup\fontsize{9}{11}\selectfont

\begin{longtable}[t]{rcccc}
\caption{\label{tab:acq-table}Quarterly event study by acquirer filing rate, 2+ unit buildings (SE in parentheses). Reference: q=-4.}\\
\toprule
q & 2plus\_highfiler\_portfolio & 2plus\_lowfiler\_portfolio & 2plus\_nonfiler\_has\_portfolio & 2plus\_singlepurchase\\
\midrule
-8 & -0.00151 (0.00250) & 0.00210 (0.00193) & -0.00151 (0.00101) & -0.00036 (0.00074)\\
-2 & -0.00269 (0.00223) & -0.00159 (0.00185) & 0.00098 (0.00118) & 0.00182 (0.00078)\\
-1 & -0.00388 (0.00240) & -0.00255 (0.00199) & -0.00135 (0.00114) & -0.00030 (0.00073)\\
0 & -0.00278 (0.00278) & -0.00327 (0.00207) & -0.00016 (0.00124) & -0.00211 (0.00071)\\
1 & -0.00054 (0.00285) & -0.00184 (0.00215) & 0.00263 (0.00136) & -0.00090 (0.00072)\\
\addlinespace
2 & 0.00070 (0.00255) & -0.00144 (0.00228) & -0.00006 (0.00126) & -0.00060 (0.00073)\\
4 & 0.00168 (0.00282) & 0.00065 (0.00255) & 0.00328 (0.00133) & 0.00094 (0.00074)\\
8 & 0.00072 (0.00304) & 0.00644 (0.00335) & 0.00191 (0.00142) & -0.00033 (0.00074)\\
\bottomrule
\end{longtable}
\endgroup{}

\subsection{5+ unit buildings}\label{unit-buildings-1}

\begingroup\fontsize{9}{11}\selectfont

\begin{longtable}[t]{lrrrrrrrr}
\caption{\label{tab:acq-5plus-stats}Acquirer classification (5+ unit buildings, at q=0).}\\
\toprule
Acquirer type & Transfers & PIDs & Owners & Total units & Mean acq rate & Mean n other & \% transfers & \% units\\
\midrule
Single-purchase & 2,586 & 2,037 & 2,569 & 53,644 & NA & 0.0 & 68.5 & 76.2\\
Non-filer (has portfolio) & 508 & 452 & 405 & 5,081 & 0.0000 & 3.1 & 13.5 & 7.2\\
Low-filer portfolio & 320 & 300 & 210 & 4,035 & 0.0209 & 16.9 & 8.5 & 5.7\\
High-filer portfolio & 362 & 305 & 244 & 7,676 & 0.1929 & 22.8 & 9.6 & 10.9\\
\bottomrule
\end{longtable}
\endgroup{}

\begin{figure}[H]

{\centering \includegraphics[width=1\linewidth,height=\textheight,keepaspectratio]{../output/figs/rtt_eviction_quarterly_by_acq_filer_5plus.png}

}

\caption{Quarterly event study by acquirer filing behavior, 5+ unit
buildings only.}

\end{figure}%

Key findings:

\begin{itemize}
\tightlist
\item
  \textbf{Single-purchase acquirers} (77.9\% of transfers, 428,657
  unique owners): These have no other matched properties. They are
  largely individual buyers (84.2\% non-corporate) with median price
  \$115,000.
\item
  \textbf{Low-filer portfolio} (5.7\%, 6,172 owners): Mean portfolio
  91.7 properties, 73\% corporate, pre-acq rate 0.0264.
\item
  \textbf{High-filer portfolio} (5.7\%, 10,870 owners): Mean portfolio
  181.4 properties, 60.2\% corporate, pre-acq rate 0.112.
\end{itemize}

\section{Portfolio Size Heterogeneity}\label{sec-portfolio}

Portfolio bins are based on the total number of \textbf{non-sheriff}
acquisitions by the buyer entity.

\subsection{2+ unit buildings}\label{unit-buildings-2}

\begin{longtable}[t]{lrrrrr}
\caption{\label{tab:port-stats-table}Portfolio size bin descriptives, 2+ unit buildings (at q=0). Bins based on non-sheriff acquisition count.}\\
\toprule
Portfolio bin & Transfers & Owners & Total units & \% corp & Mean units\\
\midrule
Single-purchase & 22,182 & 22,021 & 90,772 & 25.9 & 4.1\\
5-9 & 4,343 & 2,223 & 12,232 & 56.7 & 2.8\\
2-4 & 11,390 & 8,407 & 40,320 & 40.3 & 3.5\\
10+ & 4,258 & 1,286 & 11,690 & 74.6 & 2.7\\
Sheriff-only & 1,095 & 850 & 4,180 & 68.9 & 3.8\\
\bottomrule
\end{longtable}

\begin{figure}[H]

{\centering \includegraphics[width=1\linewidth,height=\textheight,keepaspectratio]{../output/figs/rtt_eviction_quarterly_by_portfolio.png}

}

\caption{Quarterly event study by portfolio size, 2+ unit buildings.}

\end{figure}%

\begingroup\fontsize{9}{11}\selectfont

\begin{longtable}[t]{rcccc}
\caption{\label{tab:portfolio-table}Quarterly event study by portfolio size, 2+ unit buildings. Reference: q=-4.}\\
\toprule
q & 2plus\_portfolio\_10+ & 2plus\_portfolio\_2-4 & 2plus\_portfolio\_5-9 & 2plus\_portfolio\_Single-purchase\\
\midrule
-8 & 0.00067 & -0.00124 & -0.00357 & -0.00024\\
-2 & -0.00258 & 0.00214 & -0.00105 & 0.00147\\
-1 & -0.00200 & -0.00010 & -0.00577 & -0.00015\\
0 & -0.00232 & 0.00025 & -0.00251 & -0.00262\\
1 & -0.00071 & 0.00261 & -0.00289 & -0.00109\\
\addlinespace
2 & 0.00124 & 0.00040 & -0.00181 & -0.00049\\
4 & 0.00707 & 0.00201 & 0.00118 & 0.00088\\
8 & 0.00925 & 0.00112 & 0.00168 & -0.00082\\
\bottomrule
\end{longtable}
\endgroup{}

\subsection{5+ unit buildings}\label{unit-buildings-3}

\begin{longtable}[t]{lrrrrr}
\caption{\label{tab:port-5plus-stats}Portfolio size bin descriptives, 5+ unit buildings (at q=0).}\\
\toprule
Portfolio bin & Transfers & Owners & Total units & \% corp & Mean units\\
\midrule
Single-purchase & 2,241 & 2,229 & 46,257 & 58.4 & 20.6\\
2-4 & 918 & 720 & 16,719 & 70.9 & 18.2\\
5-9 & 298 & 194 & 3,075 & 84.2 & 10.3\\
10+ & 232 & 147 & 2,478 & 78.9 & 10.7\\
Sheriff-only & 87 & 64 & 1,907 & 90.8 & 21.9\\
\bottomrule
\end{longtable}

\begin{figure}[H]

{\centering \includegraphics[width=1\linewidth,height=\textheight,keepaspectratio]{../output/figs/rtt_eviction_quarterly_by_portfolio_5plus.png}

}

\caption{Quarterly event study by portfolio size, 5+ unit buildings
only.}

\end{figure}%

\section{Transfer Pattern Descriptives}\label{sec-descriptives}

\subsection{Do high-evicting properties transfer more
often?}\label{do-high-evicting-properties-transfer-more-often}

We classify properties by their \textbf{pre-transfer filing intensity}
using \texttt{filing\_rate\_eb\_pre\_covid} from the building panel ---
the empirical Bayes smoothed annual filing rate per unit, computed from
pre-COVID data.

\textbf{Definitions:}

\begin{itemize}
\tightlist
\item
  \textbf{Above-median filing}: \texttt{filing\_rate\_eb\_pre\_covid}
  \(>\) 0.0052 (the median among properties with any filings)
\item
  \textbf{Below-median filing}: \texttt{filing\_rate\_eb\_pre\_covid}
  \(> 0\) and \(\leq\) 0.0052
\item
  \textbf{Zero-filing}: \texttt{filing\_rate\_eb\_pre\_covid} \(= 0\)
\end{itemize}

\begin{longtable}[t]{lrrrrr}
\caption{\label{tab:transfer-rate}Transfer rate by property eviction class (universe: all PIDs in bldg\_panel\_blp, 2010--2019).}\\
\toprule
Property class & N properties & N transferred & \% transferred & Mean units & Total units\\
\midrule
Above-median filing & 74,180 & 59,940 & 80.8 & 2.2 & 160,310\\
Below-median filing & 113,315 & 89,979 & 79.4 & 1.8 & 204,277\\
\bottomrule
\end{longtable}

High-evicting properties transfer at similar rates to low-evicting ones
--- approximately 80\% of properties in both categories experienced at
least one transfer during the sample period. The relevant question is
not \emph{whether} these properties transfer, but \emph{who buys them}.

\subsection{Who buys high-evicting
properties?}\label{who-buys-high-evicting-properties}

\begingroup\fontsize{9}{11}\selectfont

\begin{longtable}[t]{lrrrrrrrr}
\caption{\label{tab:buyer-by-class}Buyer characteristics by property eviction class. Above-median properties attract buyers with larger portfolios and slightly higher corporate share.}\\
\toprule
Property class & Transfers & PIDs & \% corp & Mean portfolio & Median price & Mean units & Total units & \% sheriff\\
\midrule
Above-median filing & 87,556 & 49,852 & 35.2 & 36.9 & 90,000 & 1.7 & 146,452 & 10.3\\
Below-median filing & 134,546 & 79,104 & 32.2 & 31.3 & 106,900 & 1.4 & 186,940 & 10.1\\
\bottomrule
\end{longtable}
\endgroup{}

Above-median evicting properties attract slightly more corporate buyers
(35.2\% vs 32.2\%) and buyers with somewhat larger portfolios (mean 36.9
vs 31.3 properties).

\subsection{How often do high-evicting landlords
buy?}\label{how-often-do-high-evicting-landlords-buy}

\begingroup\fontsize{9}{11}\selectfont

\begin{longtable}[t]{lrrrrrrr}
\caption{\label{tab:acq-desc-table}Acquirer filing rate descriptives. Single-purchase acquirers dominate (69\% of transfers) but portfolio acquirers control a disproportionate share of units.}\\
\toprule
Acquirer type & Transfers & \% of transfers & Unique owners & Total units & \% of units & \% corp & Mean portfolio\\
\midrule
Single-purchase & 514,970 & 77.9 & 428,657 & 212,464 & 62.3 & 15.8 & 17.5\\
Non-filer (has portfolio) & 70,606 & 10.7 & 31,061 & 55,968 & 16.4 & 43.9 & 10.0\\
Low-filer portfolio & 37,903 & 5.7 & 6,172 & 33,247 & 9.8 & 73.0 & 91.7\\
High-filer portfolio & 37,892 & 5.7 & 10,870 & 39,124 & 11.5 & 60.2 & 181.4\\
\bottomrule
\end{longtable}
\endgroup{}

\textbf{Single-purchase acquirers} account for 77.9\% of all transfers
and 62.3\% of units. \textbf{High-filer portfolio acquirers} are 5.7\%
of transfers and 11.5\% of units. They are predominantly corporate
(60.2\% corporate) with large portfolios (mean 181.4 properties).

\textbf{Low-filer portfolio acquirers} are the most concentrated group:
only 6,172 unique owners account for 5.7\% of transfers. These are
large-portfolio corporate operators (mean 91.7 properties, 73\%
corporate).

\subsection{\texorpdfstring{Cross-tab: property eviction class
\(\times\) acquirer
type}{Cross-tab: property eviction class \textbackslash times acquirer type}}\label{cross-tab-property-eviction-class-times-acquirer-type}

\begin{longtable}[t]{llrrr}
\caption{\label{tab:cross-tab}Who buys which properties? Cross-tab of property eviction class by acquirer type.}\\
\toprule
Property class & Acquirer type & Transfers & Total units & \% of class\\
\midrule
Above-median filing & High-filer portfolio & 12,887 & 21,979 & 14.7\\
Above-median filing & Low-filer portfolio & 9,234 & 13,592 & 10.5\\
Above-median filing & Non-filer (has portfolio) & 15,353 & 20,934 & 17.5\\
Above-median filing & Single-purchase & 50,081 & 89,944 & 57.2\\
Below-median filing & High-filer portfolio & 13,776 & 16,853 & 10.2\\
\addlinespace
Below-median filing & Low-filer portfolio & 14,881 & 19,414 & 11.1\\
Below-median filing & Non-filer (has portfolio) & 25,659 & 34,249 & 19.1\\
Below-median filing & Single-purchase & 80,229 & 116,421 & 59.6\\
\bottomrule
\end{longtable}

The cross-tab reveals \textbf{sorting}: high-filer portfolio acquirers
account for 14.7\% of above-median property transfers but only 10.2\% of
below-median ones. Conversely, single-purchase acquirers dominate
below-median properties (59.6\%) more than above-median ones (57.2\%).

\section{Summary}\label{summary}

The quarterly event study resolves the pre-trend puzzle from the annual
analysis and provides rich heterogeneity:

\begin{enumerate}
\def\labelenumi{\arabic{enumi}.}
\item
  \textbf{The annual \(t=-1\) artifact is explained.} Filing rates peak
  at \(q = -2\), then drop sharply at \(q = -1\) and bottom at
  \(q = 0\).
\item
  \textbf{The pre-transfer dip is a small-building vacancy effect.} For
  10+ unit buildings, filing rates are stable through the transition.
\item
  \textbf{The post-transfer increase is real and operational.} It
  appears across building sizes and is larger for corporate buyers
  (among 5+ unit buildings).
\item
  \textbf{High-filer portfolio acquirers drive the effect.} The 5.7\% of
  transfers going to high-filer portfolios show the largest
  post-transfer filing increases; single-purchase acquirers show smaller
  or no increases.
\item
  \textbf{Sorting exists.} High-filer acquirers disproportionately buy
  already-high-evicting properties, consistent with a market for
  high-eviction buildings.
\end{enumerate}




\end{document}
