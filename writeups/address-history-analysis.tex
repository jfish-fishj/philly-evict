% Options for packages loaded elsewhere
% Options for packages loaded elsewhere
\PassOptionsToPackage{unicode}{hyperref}
\PassOptionsToPackage{hyphens}{url}
\PassOptionsToPackage{dvipsnames,svgnames,x11names}{xcolor}
%
\documentclass[
  11pt,
]{article}
\usepackage{xcolor}
\usepackage[margin=1in]{geometry}
\usepackage{amsmath,amssymb}
\setcounter{secnumdepth}{5}
\usepackage{iftex}
\ifPDFTeX
  \usepackage[T1]{fontenc}
  \usepackage[utf8]{inputenc}
  \usepackage{textcomp} % provide euro and other symbols
\else % if luatex or xetex
  \usepackage{unicode-math} % this also loads fontspec
  \defaultfontfeatures{Scale=MatchLowercase}
  \defaultfontfeatures[\rmfamily]{Ligatures=TeX,Scale=1}
\fi
\usepackage{lmodern}
\ifPDFTeX\else
  % xetex/luatex font selection
\fi
% Use upquote if available, for straight quotes in verbatim environments
\IfFileExists{upquote.sty}{\usepackage{upquote}}{}
\IfFileExists{microtype.sty}{% use microtype if available
  \usepackage[]{microtype}
  \UseMicrotypeSet[protrusion]{basicmath} % disable protrusion for tt fonts
}{}
\makeatletter
\@ifundefined{KOMAClassName}{% if non-KOMA class
  \IfFileExists{parskip.sty}{%
    \usepackage{parskip}
  }{% else
    \setlength{\parindent}{0pt}
    \setlength{\parskip}{6pt plus 2pt minus 1pt}}
}{% if KOMA class
  \KOMAoptions{parskip=half}}
\makeatother
% Make \paragraph and \subparagraph free-standing
\makeatletter
\ifx\paragraph\undefined\else
  \let\oldparagraph\paragraph
  \renewcommand{\paragraph}{
    \@ifstar
      \xxxParagraphStar
      \xxxParagraphNoStar
  }
  \newcommand{\xxxParagraphStar}[1]{\oldparagraph*{#1}\mbox{}}
  \newcommand{\xxxParagraphNoStar}[1]{\oldparagraph{#1}\mbox{}}
\fi
\ifx\subparagraph\undefined\else
  \let\oldsubparagraph\subparagraph
  \renewcommand{\subparagraph}{
    \@ifstar
      \xxxSubParagraphStar
      \xxxSubParagraphNoStar
  }
  \newcommand{\xxxSubParagraphStar}[1]{\oldsubparagraph*{#1}\mbox{}}
  \newcommand{\xxxSubParagraphNoStar}[1]{\oldsubparagraph{#1}\mbox{}}
\fi
\makeatother


\usepackage{longtable,booktabs,array}
\usepackage{calc} % for calculating minipage widths
% Correct order of tables after \paragraph or \subparagraph
\usepackage{etoolbox}
\makeatletter
\patchcmd\longtable{\par}{\if@noskipsec\mbox{}\fi\par}{}{}
\makeatother
% Allow footnotes in longtable head/foot
\IfFileExists{footnotehyper.sty}{\usepackage{footnotehyper}}{\usepackage{footnote}}
\makesavenoteenv{longtable}
\usepackage{graphicx}
\makeatletter
\newsavebox\pandoc@box
\newcommand*\pandocbounded[1]{% scales image to fit in text height/width
  \sbox\pandoc@box{#1}%
  \Gscale@div\@tempa{\textheight}{\dimexpr\ht\pandoc@box+\dp\pandoc@box\relax}%
  \Gscale@div\@tempb{\linewidth}{\wd\pandoc@box}%
  \ifdim\@tempb\p@<\@tempa\p@\let\@tempa\@tempb\fi% select the smaller of both
  \ifdim\@tempa\p@<\p@\scalebox{\@tempa}{\usebox\pandoc@box}%
  \else\usebox{\pandoc@box}%
  \fi%
}
% Set default figure placement to htbp
\def\fps@figure{htbp}
\makeatother





\setlength{\emergencystretch}{3em} % prevent overfull lines

\providecommand{\tightlist}{%
  \setlength{\itemsep}{0pt}\setlength{\parskip}{0pt}}



 


\usepackage{booktabs}
\usepackage{longtable}
\usepackage{array}
\usepackage{multirow}
\usepackage{wrapfig}
\usepackage{float}
\usepackage{colortbl}
\usepackage{pdflscape}
\usepackage{tabu}
\usepackage{threeparttable}
\usepackage{threeparttablex}
\usepackage[normalem]{ulem}
\usepackage{makecell}
\usepackage{xcolor}
\usepackage{booktabs}
\usepackage{dcolumn}
\usepackage{longtable}
\usepackage{adjustbox}
\makeatletter
\@ifpackageloaded{caption}{}{\usepackage{caption}}
\AtBeginDocument{%
\ifdefined\contentsname
  \renewcommand*\contentsname{Table of contents}
\else
  \newcommand\contentsname{Table of contents}
\fi
\ifdefined\listfigurename
  \renewcommand*\listfigurename{List of Figures}
\else
  \newcommand\listfigurename{List of Figures}
\fi
\ifdefined\listtablename
  \renewcommand*\listtablename{List of Tables}
\else
  \newcommand\listtablename{List of Tables}
\fi
\ifdefined\figurename
  \renewcommand*\figurename{Figure}
\else
  \newcommand\figurename{Figure}
\fi
\ifdefined\tablename
  \renewcommand*\tablename{Table}
\else
  \newcommand\tablename{Table}
\fi
}
\@ifpackageloaded{float}{}{\usepackage{float}}
\floatstyle{ruled}
\@ifundefined{c@chapter}{\newfloat{codelisting}{h}{lop}}{\newfloat{codelisting}{h}{lop}[chapter]}
\floatname{codelisting}{Listing}
\newcommand*\listoflistings{\listof{codelisting}{List of Listings}}
\makeatother
\makeatletter
\makeatother
\makeatletter
\@ifpackageloaded{caption}{}{\usepackage{caption}}
\@ifpackageloaded{subcaption}{}{\usepackage{subcaption}}
\makeatother
\usepackage{bookmark}
\IfFileExists{xurl.sty}{\usepackage{xurl}}{} % add URL line breaks if available
\urlstyle{same}
\hypersetup{
  pdftitle={Address History Analysis: Eviction Persistence and Mover Trajectories},
  pdfauthor={Philly Evictions Project},
  colorlinks=true,
  linkcolor={blue},
  filecolor={Maroon},
  citecolor={Blue},
  urlcolor={Blue},
  pdfcreator={LaTeX via pandoc}}


\title{Address History Analysis: Eviction Persistence and Mover
Trajectories}
\author{Philly Evictions Project}
\date{2026-02-20}
\begin{document}
\maketitle

\renewcommand*\contentsname{Table of contents}
{
\hypersetup{linkcolor=}
\setcounter{tocdepth}{3}
\tableofcontents
}

\section{Overview}\label{overview}

This document summarizes the address history analysis from
\texttt{r/address-history-analysis.R}. The script uses the InfoUSA
household panel to track tenants across moves and test whether exposure
to high-eviction buildings channels households into worse neighborhoods.

\textbf{Key questions:}

\begin{enumerate}
\def\labelenumi{\arabic{enumi}.}
\tightlist
\item
  Do tenants from high-eviction buildings end up in other high-eviction
  buildings? (Eviction persistence)
\item
  What are the demographic characteristics of movers vs.~stayers?
\item
  Do tenants who move \emph{into} high-eviction buildings come from
  different neighborhoods? (Inflow)
\item
  Where do tenants go \emph{after} leaving high-eviction buildings?
  (Outflow)
\item
  Across a 3-move trajectory (origin \(\to\) middle \(\to\)
  destination), does passing through a high-evicting building shift
  tenants toward higher-eviction neighborhoods? (Trajectory analysis)
\end{enumerate}

\textbf{Unit of analysis:} Household (familyid) \(\times\) year panel,
with building-level (PID) characteristics merged in.

\textbf{Key data inputs:}

\begin{itemize}
\tightlist
\item
  \texttt{infousa\_cleaned} --- household-year panel with addresses
\item
  \texttt{infousa\_address\_xwalk} --- links InfoUSA addresses to parcel
  IDs
\item
  \texttt{bldg\_panel\_blp} --- building characteristics, filing rates,
  rents, occupancy
\item
  \texttt{infousa\_race\_imputed\_person},
  \texttt{infousa\_gender\_imputed\_person} --- demographic imputations
\end{itemize}

\section{Tracking Descriptives}\label{tracking-descriptives}

\subsection{Movers vs.~Stayers}\label{movers-vs.-stayers}

The script identifies rental households and classifies them as
\emph{movers} (2+ addresses over the panel) or \emph{stayers} (1
address). Summary statistics are computed for each group.

\begin{table}[htbp]
\centering
\centering
\caption{Tracking Descriptives: Movers vs.\ Stayers}
\label{tab:tracking_descriptives}

\begin{adjustbox}{max width=\textwidth}
\begin{tabular}{lrrr}
\toprule
 & All Rental HHs & Stayers (1 address) & Movers (2+ addresses) \\
\midrule
N HH-Years & 6,563,638 & 5,253,915 & 1,309,723 \\
N Unique HHs & 1,641,033 & 1,515,419 &   125,614 \\
Mean Addresses per HH & 1.2 & 1 & 3.66 \\
Mean Years Observed & 3.69 & 3.47 & 6.34 \\
\% Black (mean posterior) & 36.7 & 36.6 & 37.2 \\
\% White (mean posterior) & 54 & 53.7 & 55.9 \\
\% Hispanic (mean posterior) & 6.4 & 6.6 & 5.3 \\
\% Female HoH & 55.1 & 54.1 & 61.1 \\
Mean Filing Rate & 0.0646 & 0.0673 & 0.0537 \\
Mean Building Size (units) & 29.9 & 30.3 & 28.3 \\
Mean Log Rent & 7.083 & 7.064 & 7.153 \\
N Unique Tracts & 379 & 379 & 372 \\
\bottomrule
\end{tabular}

\end{adjustbox}
\end{table}

\section{Eviction Persistence
Regressions}\label{eviction-persistence-regressions}

\subsection{Setup}\label{setup}

The core persistence analysis asks: conditional on a household
\emph{moving}, does the eviction filing rate at their \textbf{previous}
building predict the filing rate at their \textbf{current} building?

\textbf{Sample:} Rental-to-rental movers (both origin and current
address are rental buildings). Filing rates are capped at reasonable
bounds to exclude outliers.

\subsection{Regression Specifications}\label{regression-specifications}

\begin{longtable}[]{@{}
  >{\raggedright\arraybackslash}p{(\linewidth - 8\tabcolsep) * \real{0.1373}}
  >{\raggedright\arraybackslash}p{(\linewidth - 8\tabcolsep) * \real{0.0980}}
  >{\raggedright\arraybackslash}p{(\linewidth - 8\tabcolsep) * \real{0.0980}}
  >{\raggedright\arraybackslash}p{(\linewidth - 8\tabcolsep) * \real{0.2941}}
  >{\raggedright\arraybackslash}p{(\linewidth - 8\tabcolsep) * \real{0.3725}}@{}}
\toprule\noalign{}
\begin{minipage}[b]{\linewidth}\raggedright
Model
\end{minipage} & \begin{minipage}[b]{\linewidth}\raggedright
LHS
\end{minipage} & \begin{minipage}[b]{\linewidth}\raggedright
RHS
\end{minipage} & \begin{minipage}[b]{\linewidth}\raggedright
Fixed Effects
\end{minipage} & \begin{minipage}[b]{\linewidth}\raggedright
Sample restriction
\end{minipage} \\
\midrule\noalign{}
\endhead
\bottomrule\noalign{}
\endlastfoot
m1 & Filing rate & Prev. filing rate & None & Rate \(\leq 1\) \\
m2 & Filing rate & Prev. filing rate & Unit bins + BG (current \& prev)
& Rate \(\leq 1\) \\
m3 & Filing rate & Prev. filing rate + imputed rent & Unit bins + BG &
Rate \(\leq 0.5\) \\
m4 & Filing rate & Prev. filing rate + actual rent & Unit bins + BG &
Rate \(\leq 1\) \\
m5\_no\_rent & High-filer (\$\textgreater\$10\%) & Prev. high-filer &
Unit bins + BG & Rate \(\leq 0.5\) \\
m5 & High-filer & Prev. high-filer + rent & Unit bins + BG & Rate
\(\leq 0.5\), multi-unit \\
m\_rent & Log rent & Prev. filing rate + prev. rent + filing rate & Unit
bins + BG & Rate \(\leq 0.5\) \\
\end{longtable}

All models cluster standard errors by PID.

\begin{table}[htbp]
\centering
\caption{\label{tab:evict_persist} Effect of Previous Eviction Filing Rate on Current Filing Rate}
\centering

\begin{adjustbox}{max width=\textwidth}
   \begin{tabular}{lccc}
      \tabularnewline \midrule \midrule
      Dependent Variable: & \multicolumn{3}{c}{filing\_rate\_raw}\\
      Model:                        & (1)            & (2)            & (3)\\  
      \midrule
      \emph{Variables}\\
      Previous Eviction Filing Rate & 0.0992$^{***}$ & 0.0795$^{***}$ & 0.1950$^{***}$\\   
                                    & (0.0040)       & (0.0042)       & (0.0144)\\   
      Rent (current)                &                &                & -0.0914$^{***}$\\   
                                    &                &                & (0.0073)\\   
      Rent (previous)               &                &                & -0.0234$^{***}$\\   
                                    &                &                & (0.0063)\\   
      \midrule
      \emph{Fixed-effects}\\
      Units (current)               &                & Yes            & Yes\\  
      Units (previous)              &                & Yes            & Yes\\  
      Block Group (current)         &                & Yes            & Yes\\  
      Block Group (previous)        &                & Yes            & Yes\\  
      \midrule
      \emph{Fit statistics}\\
      Observations                  & 144,254        & 120,232        & 15,716\\  
      \midrule \midrule
      \multicolumn{4}{l}{\emph{Clustered (pid) standard-errors in parentheses}}\\
      \multicolumn{4}{l}{\emph{Signif. Codes: ***: 0.01, **: 0.05, *: 0.1}}\\
   \end{tabular}

\end{adjustbox}
\end{table}

\section{Inflow and Outflow Analysis}\label{inflow-and-outflow-analysis}

\subsection{Leave-One-Out Tract Eviction
Rate}\label{leave-one-out-tract-eviction-rate}

For each building, the LOO tract eviction rate is:

\[
\text{LOO}_{i} = \frac{\sum_{j \in \text{tract}(i), j \neq i} \text{filings}_{j}}{\sum_{j \in \text{tract}(i), j \neq i} \text{units}_{j}}
\]

This measures the eviction environment of the building's neighborhood,
excluding the building's own contribution.

\subsection{Eviction Intensity Bins}\label{eviction-intensity-bins}

Buildings are classified using the empirical Bayes eviction filing rate
(pre-COVID), with true never-filers (zero pre-2019 filings) separated
from the low-rate category:

\begin{longtable}[]{@{}ll@{}}
\toprule\noalign{}
Bin & Definition \\
\midrule\noalign{}
\endhead
\bottomrule\noalign{}
\endlastfoot
No filings & \texttt{total\_filings\_pre2019\ ==\ 0} \\
(0--5\%{]} & EB rate \(\leq 0.05\), at least 1 filing \\
(5--10\%{]} & EB rate \(\in (0.05, 0.10]\) \\
(10--20\%{]} & EB rate \(\in (0.10, 0.20]\) \\
20\%+ & EB rate \(> 0.20\) \\
\end{longtable}

\subsection{Inflow: Where Do Tenants at High-Eviction Buildings Come
From?}\label{inflow-where-do-tenants-at-high-eviction-buildings-come-from}

\begin{longtable}[t]{llrrrrrr}
\caption{\label{tab:inflow-table}Origin characteristics of movers, by destination eviction intensity}\\
\toprule
Destination bin & N moves & Origin \% Black & Dest \% Black & Origin LOO evict & Dest LOO evict & Mover \% Black & Mover \% Female\\
\midrule
No filings & 63,790 & 0.358 & 0.368 & 0.0346 & 0.0346 & 0.290 & 0.590\\
(0-5\%] & 15,813 & 0.289 & 0.260 & 0.0305 & 0.0287 & 0.189 & 0.588\\
(5-10\%] & 18,113 & 0.484 & 0.489 & 0.0462 & 0.0487 & 0.450 & 0.626\\
(10-20\%] & 14,633 & 0.511 & 0.513 & 0.0496 & 0.0521 & 0.481 & 0.643\\
20\%+ & 10,578 & 0.565 & 0.571 & 0.0557 & 0.0612 & 0.529 & 0.656\\
\bottomrule
\end{longtable}

\subsection{Outflow: Where Do Tenants From High-Eviction Buildings
Go?}\label{outflow-where-do-tenants-from-high-eviction-buildings-go}

\begin{longtable}[t]{llrrrrr}
\caption{\label{tab:outflow-table}Destination characteristics of movers, by origin eviction intensity}\\
\toprule
Origin bin & N moves & Origin LOO evict & Dest LOO evict & \$\textbackslash{}Delta\$ LOO tract & Mover \% Black & Mover \% Female\\
\midrule
No filings & 59,788 & 0.0341 & 0.0368 & 0.0026 & 0.299 & 0.588\\
(0-5\%] & 16,473 & 0.0248 & 0.0267 & 0.0019 & 0.177 & 0.578\\
(5-10\%] & 19,336 & 0.0465 & 0.0459 & -0.0006 & 0.422 & 0.631\\
(10-20\%] & 15,838 & 0.0508 & 0.0497 & -0.0011 & 0.465 & 0.644\\
20\%+ & 11,295 & 0.0604 & 0.0557 & -0.0048 & 0.527 & 0.662\\
\bottomrule
\end{longtable}

\section{Three-Move Trajectory
Analysis}\label{three-move-trajectory-analysis}

\subsection{Design}\label{design}

The trajectory analysis tracks households across three consecutive
residential spells:

\[
\text{Origin} \xrightarrow{\text{move 1}} \text{Middle} \xrightarrow{\text{move 2}} \text{Destination}
\]

The key question: conditional on origin neighborhood, does the eviction
intensity of the \textbf{middle} building predict worse neighborhood
outcomes at the \textbf{destination}?

The outcome is the change in LOO tract eviction rate from origin to
destination: \[
\Delta\text{LOO} = \text{LOO}_{\text{dest}} - \text{LOO}_{\text{origin}}
\]

\subsection{Descriptive Summary}\label{descriptive-summary}

\begin{longtable}[t]{llrrrrr}
\caption{\label{tab:trajectory-table}Three-move trajectories by middle building eviction intensity}\\
\toprule
Middle bin & N trajectories & Origin LOO & Dest LOO & \$\textbackslash{}Delta\$ LOO & Origin EB rate & Dest EB rate\\
\midrule
No filings & 100,463 & 0.0408 & 0.0410 & 2e-04 & 0.0396 & 0.0430\\
(0-5\%] & 6,561 & 0.0358 & 0.0359 & 2e-04 & 0.0491 & 0.0501\\
(5-10\%] & 19,254 & 0.0494 & 0.0498 & 5e-04 & 0.0575 & 0.0625\\
(10-20\%] & 13,959 & 0.0524 & 0.0527 & 3e-04 & 0.0685 & 0.0766\\
20\%+ & 6,632 & 0.0570 & 0.0573 & 2e-04 & 0.0991 & 0.1163\\
\bottomrule
\end{longtable}

\subsection{Trajectory Regressions}\label{trajectory-regressions}

Three specifications test whether the middle building's eviction
intensity predicts the origin-to-destination change in neighborhood
eviction environment:

\begin{longtable}[]{@{}
  >{\raggedright\arraybackslash}p{(\linewidth - 8\tabcolsep) * \real{0.1395}}
  >{\raggedright\arraybackslash}p{(\linewidth - 8\tabcolsep) * \real{0.1163}}
  >{\raggedright\arraybackslash}p{(\linewidth - 8\tabcolsep) * \real{0.1163}}
  >{\raggedright\arraybackslash}p{(\linewidth - 8\tabcolsep) * \real{0.3488}}
  >{\raggedright\arraybackslash}p{(\linewidth - 8\tabcolsep) * \real{0.2791}}@{}}
\toprule\noalign{}
\begin{minipage}[b]{\linewidth}\raggedright
Spec
\end{minipage} & \begin{minipage}[b]{\linewidth}\raggedright
LHS
\end{minipage} & \begin{minipage}[b]{\linewidth}\raggedright
RHS
\end{minipage} & \begin{minipage}[b]{\linewidth}\raggedright
Fixed Effects
\end{minipage} & \begin{minipage}[b]{\linewidth}\raggedright
Clustering
\end{minipage} \\
\midrule\noalign{}
\endhead
\bottomrule\noalign{}
\endlastfoot
(A) & \(\Delta\) LOO tract & \texttt{i(mid\_evict\_bin)} & None & Origin
BG \\
(B) & \(\Delta\) LOO tract & \texttt{i(mid\_evict\_bin)} & Origin tract
& Origin BG \\
(C) & Dest LOO tract (level) & \texttt{i(mid\_evict\_bin)} & Origin
tract & Origin BG \\
\end{longtable}

Reference category: (0--5\%{]}.

\begin{adjustbox}{max width=\textwidth}
\begingroup
\centering
\begin{tabular}{lccc}
   \tabularnewline \midrule \midrule
   Dependent Variables: & \multicolumn{2}{c}{delta\_loo\_tract} & dest\_loo\_tract\\
                                             & Delta LOO Tract       & Delta + Orig FE       & Dest Level + Orig FE \\   
   Model:                                    & (1)                   & (2)                   & (3)\\  
   \midrule
   \emph{Variables}\\
   Constant                                  & 0.0001                &                       &   \\   
                                             & (0.0004)              &                       &   \\   
   mid\_evict\_bin $=$ Nofilings0-5\%]")     & $8.46\times 10^{-5}$  & $7.33\times 10^{-5}$  & 0.0003\\   
                                             & (0.0003)              & (0.0003)              & (0.0003)\\   
   mid\_evict\_bin $=$ (5-10\%]0-5\%]")      & 0.0003                & 0.0010$^{***}$        & 0.0011$^{***}$\\   
                                             & (0.0003)              & (0.0003)              & (0.0003)\\   
   mid\_evict\_bin $=$ (10-20\%]0-5\%]")     & 0.0002                & 0.0012$^{***}$        & 0.0013$^{***}$\\   
                                             & (0.0004)              & (0.0003)              & (0.0003)\\   
   mid\_evict\_bin $=$ 20\%+0-5\%]")         & $3.31\times 10^{-5}$  & 0.0020$^{***}$        & 0.0015$^{***}$\\   
                                             & (0.0005)              & (0.0004)              & (0.0004)\\   
   \midrule
   \emph{Fixed-effects}\\
   orig\_tract                               &                       & Yes                   & Yes\\  
   \midrule
   \emph{Fit statistics}\\
   Observations                              & 146,636               & 146,634               & 146,634\\  
   R$^2$                                     & $2.61\times 10^{-5}$  & 0.11608               & 0.73635\\  
   Within R$^2$                              &                       & 0.00116               & 0.00074\\  
   \midrule \midrule
   \multicolumn{4}{l}{\emph{Clustered (orig\_GEOID) standard-errors in parentheses}}\\
   \multicolumn{4}{l}{\emph{Signif. Codes: ***: 0.01, **: 0.05, *: 0.1}}\\
\end{tabular}
\par\endgroup

\end{adjustbox}

\section{Interpretation}\label{interpretation}

\textbf{Eviction persistence:} The positive coefficient on previous
filing rate indicates that tenants who leave high-eviction buildings
tend to end up in other high-eviction buildings, even after controlling
for block group and building size. This is not simply a neighborhood
effect --- it persists within block groups.

\textbf{Inflow/Outflow:} High-eviction buildings draw tenants from
neighborhoods that are already higher-eviction, and their departing
tenants move to similar or worse neighborhoods. The LOO tract eviction
rate captures the neighborhood environment excluding the focal building.

\textbf{Trajectories:} The key finding from the trajectory regressions
is specification (B): after conditioning on origin tract, passing
through a high-eviction middle building predicts a larger increase in
neighborhood eviction rate from origin to destination. This suggests
that high-eviction buildings act as a \emph{channeling mechanism},
directing tenants toward worse eviction environments relative to their
starting point.




\end{document}
