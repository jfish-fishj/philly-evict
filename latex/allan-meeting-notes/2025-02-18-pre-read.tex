\documentclass[10pt, xcolor=dvipsnames]{beamer}

% There are many different themes available for Beamer. A comprehensive
% list with examples is given here:
% http://deic.uab.es/~iblanes/beamer_gallery/index_by_theme.html
% You can uncomment the themes below if you would like to use a different
% one:
%\usetheme{AnnArbor}
%\usetheme{Antibes}
%\usetheme{Bergen}
%\usetheme{Berkeley}
%\usetheme{Berlin}
%\usetheme{Boadilla}
%\usetheme{boxes}
%\usetheme{CambridgeUS}
%\usetheme{Copenhagen}
%\usetheme{Darmstadt}
%\usetheme{default}
%\usetheme{Frankfurt}
%\usetheme{Goettingen}
%\usetheme{Hannover}
%\usetheme{Ilmenau}
%\usetheme{JuanLesPins}
%\usetheme{Luebeck}
%\usetheme{Frankfurt}
\usetheme{Madrid}
%\usetheme{Malmoe}
%\usetheme{Marburg}
%\usetheme{Montpellier}
%\usetheme{PaloAlto}
%\usetheme{Pittsburgh}
%\usetheme{Rochester}
%\usetheme{Singapore}
%\usetheme{Szeged}
%\usetheme{Warsaw}

\usepackage[utf8]{inputenc}
\usepackage[english]{babel}
\usepackage{ragged2e}
\usepackage{bbding}
%\usepackage{enumitem}
\usepackage{mathtools}
\usepackage{indentfirst}
\usepackage{graphicx}
\usepackage{float}
\usepackage{hyperref}
\usepackage{mathtools}
\usepackage{preview}
\usepackage{xcolor}
\usepackage{color}
\usepackage{listings}
\usepackage{float}
\usepackage[caption = false]{subfig}
\usepackage{pdfpages}
\usepackage{multirow}
\usepackage{array}
\usepackage{makecell}
\usepackage{bm}
\usepackage{caption}
\usepackage{cancel}
\usepackage{anyfontsize}
\usepackage{etoolbox}
\usepackage{amsmath}
\usepackage{amssymb}
\usepackage{mathtools}
\usepackage{natbib}
\usepackage[flushleft]{threeparttable}
\usepackage{booktabs}
\usepackage{caption}
\usepackage{adjustbox}
\usepackage{appendixnumberbeamer}
\usepackage{pifont}
\usepackage{amsmath}
\usepackage{amssymb}
\usepackage[percent]{overpic}



\setbeamercolor{titlelike}{parent=structure}
\definecolor{UBCblue}{rgb}{0.04706, 0.13725, 0.32}
\colorlet{UBCblue2}{UBCblue!70!white}
\usecolortheme[named=UBCblue]{structure}

\makeatletter
\setbeamertemplate{footline}
{
  \leavevmode%
  \hbox{%
  \begin{beamercolorbox}[wd=.4\paperwidth,ht=2.25ex,dp=1ex,center]{author in head/foot}%
    \usebeamerfont{author in head/foot} \insertshortauthor %\hspace*{1em}(\insertshortinstitute)
  \end{beamercolorbox}%
  \begin{beamercolorbox}[wd=.5\paperwidth,ht=2.25ex,dp=1ex,center]{title in head/foot}%
    \usebeamerfont{title in head/foot} \insertshorttitle
  \end{beamercolorbox}%
  \begin{beamercolorbox}[wd=.1\paperwidth,ht=2.25ex,dp=1ex,center]{date in head/foot}%
    \usebeamerfont{date in head/foot}
    \insertframenumber{} / \inserttotalframenumber\hspace*{2ex} 
  \end{beamercolorbox}}%
  \vskip0pt%
}
\makeatother

\renewcommand{\arraystretch}{1.2}
\renewcommand{\raggedright}{\leftskip=0pt \rightskip=0pt plus 0cm}
\newcolumntype{C}[1]{>{\centering\let\newline\\\arraybackslash\hspace{0pt}}m{#1}}

\hypersetup{
    colorlinks=true,
    linkcolor=UBCblue,
    citecolor=UBCblue,
    filecolor=magenta,      
    urlcolor=blue,
    allcolors=.
}
\setbeamercolor{button}{bg=UBCblue2,fg=white}
\newcommand\fnote[1]{\captionsetup{font=tiny}\caption*{#1}}
\newcommand\fnotev[1]{\captionsetup{font=scriptsize}\caption*{#1}}
\setbeamertemplate{caption}[numbered]


%\justifying
\urlstyle{same}
%\usefonttheme{serif}

%------------------------
%------------------------

\date{}

%------------------------
%------------------------
%----------------------------------------------------------------------------------------
%	TITLE PAGE
%----------------------------------------------------------------------------------------------------------------
%------------------------

\title[Landlord Responses to Changes in Tenant Protections]{Are Slumlords Necessary: \\Landlord Responses to Changes in Tenant Protections} % The short title appears at the bottom of every slide, the full title is only on the title page
\author[Joe Fish]{Joe Fish}


\begin{document}

\begin{frame}
\titlepage % Print the title page as the first slide
\end{frame}


\begin{frame}{Research Question and Motivation}
   \textbf{ What does a city do about slumlords?}
   \begin{itemize}
    \item Most US cities have awful tenant protections
    \item Tenants perceived as risky face large barriers to finding housing. Lots of rental listings will have "no eviction, no conviction" clauses, effectively locking them out of large parts of the rental market. This is true for private and public landlords.
    \item Landlords that rent to risky tenants are providing a service, these landlords are also much more likely to be predatory.
    \item When advocacy groups push for stronger tenant protections, landlords (public and private) argue that increased tenant protections would make them unable to be effective landlords. Thus, it's ex ante, unclear what the best way to regulate low income landlords is.
\end{itemize}
\end{frame}

\begin{frame}{Stylized facts}
    \textbf{Evictions in America are heavily concentrated:}
\begin{itemize}
    \item They are concentrated within certain cities (poorer, Blacker ones)
    \item Within cities, they are concentrated within certain neighborhoods (poorer, Blacker ones)
    \item Within neighborhoods, they are concentrated within certain buildings and within specific landlords 
\end{itemize} 
    \textbf{Rents in very low quality units are surprisingly high }
\end{frame}

% \begin{frame}{Motivation}
% Econ probably coalescing on evictions as a "natural" part of low income rental market
%         \begin{itemize}
%             \item most evictions are for nonpayment of rent, nonpayment is pretty common
%         \end{itemize}
%         \bigskip
% Implications:
%         \begin{itemize}
%             \item 
%         \item  limited role for policy (tenant payments; renters' insurance; expanded vouchers)
%         \item  filing fee increases, right to council, all ineffective and maybe negatives
%         \item higher rents for lower quality units mostly reflecting risk premia
%         \end{itemize}
    
% \end{frame}

\begin{frame}{Someone wrote the paper you wanted me to write}
 paper by Winnie van dijk basically does the paper you wanted. They partnered with a low income landlord in the Midwest and have lease-ledger data 
\begin{itemize}
        \item \url{https://www.dropbox.com/scl/fi/ttfuwljadd6iwbkmwf45g/Nonpayment_and_Eviction.pdf?rlkey=v872x6cavh8cjzybzj6pg3z2f&e=4&st=6eumcbqm&dl=0}
    
\end{itemize}
\textbf{stylized facts}
    \begin{itemize}
        \item most evictions for non-payment of rent; similar two months backrent median to other research
        \item non-payment of rent is very common
        \item landlords tolerate non-payment of rent because subsequent repayment is common enough
    \end{itemize}

\end{frame}

\begin{frame}{Stylized Model}
    \textbf{stylized model}
    \begin{itemize}
        \item tenants draw from CDF of probabilities of payment in a month (think income shock)
        \item based on draw, tenants do / don't pay
        \item landlords form posteriors over tenant's ability to pay in the future
        \item landlords have value function based on tenants observed state;  decide whether to evict if some threshold is crossed
    \end{itemize}
    \textbf{model delivers}
    \begin{itemize}
        \item common anti-eviction policies only delay eviction by adding costs to landlord side (IMO inconsistent w/ evidence of these policies)
        \item relative ineffectiveness for some common anti-eviction policies (filing fees, rent support, right to counsel). 
    \end{itemize}
    Companion paper also shows pass through of tenant protections onto prices
        \begin{itemize}
            \item \url{https://www.dropbox.com/scl/fi/hhc3pipg9x0jwtq5adjlm/RTC.pdf?rlkey=qky6pxy9oqe3ei39c3hm0fhjw&e=1&st=8byxwhnq&dl=0}
        \end{itemize}
\end{frame}

\begin{frame}{Ideas I've been toying with}
In loose order of what I think would be a successful JMP (ignoring data constraints for now)
\bigskip
    \begin{itemize}
        \item Two-sided model of bargaining where evictions serve as a way for landlords to get rid of delinquent tenants and as a way to be predatory
        \begin{itemize}
            \item Hope for this is that the evictions are a general case of a bargaining model where we might think power dynamics and undesirable behavior cut in multiple ways
        \end{itemize}
        \item Market power in the low income rental market with changes to tenant protections
        \begin{itemize}
            \item Intuition is that rent premia for high evicting units reflects risk premia of renting to high delinquency tenants, but also leads to very concentrated submarkets
            \item Possible add-on is that policies that discourage landlords from renting to marginal tenants make low income housing shortages worse
        \end{itemize}
    \end{itemize}
    
\end{frame}

\begin{frame}{Things I'd like help on}
\begin{itemize}
    \item which of those (if any) sound most fruitful to work on
    \item fleshing out #1 as a model
    \item Misc. field paper thoughts as time permits
\end{itemize}  
\end{frame}
    

% \begin{frame}{Things I've been working on / need help on}
% \begin{itemize}
%     \item What's the econ?
%     \item Descriptive on how different landlords use evictions
% \end{itemize}
    
% \end{frame}

\end{document}
