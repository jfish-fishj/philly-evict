\documentclass[10pt, xcolor=dvipsnames]{beamer}

\usetheme{metropolis}
\usepackage[utf8]{inputenc}
\usepackage[english]{babel}
\usepackage{ragged2e}
\usepackage{bbding}
%\usepackage{enumitem}
\usepackage{mathtools}
\usepackage{indentfirst}
\usepackage{graphicx}
\usepackage{float}
\usepackage{hyperref}
\usepackage{mathtools}
\usepackage{preview}
\usepackage{xcolor}
\usepackage{color}
\usepackage{listings}
\usepackage{float}
\usepackage[caption = false]{subfig}
\usepackage{pdfpages}
\usepackage{multirow}
\usepackage{array}
\usepackage{makecell}
\usepackage{bm}
\usepackage{caption}
\usepackage{cancel}
\usepackage{anyfontsize}
\usepackage{etoolbox}
\usepackage{mathtools}

\usepackage[flushleft]{threeparttable}
\usepackage{booktabs}
\usepackage{caption}
\usepackage{adjustbox}
\usepackage{appendixnumberbeamer}
\usepackage{pifont}
\usepackage{amsmath}
\usepackage{amssymb}
\usepackage[percent]{overpic}

%\usepackage{biblatex}
\usepackage[backend=biber,
style=authoryear,
citestyle=authoryear]{biblatex} 
\addbibresource{latex/bibliography.bib}

% \setbeamercolor{titlelike}{parent=structure}
% \definecolor{UBCblue}{rgb}{0.04706, 0.13725, 0.32}
% \colorlet{UBCblue2}{UBCblue!70!white}
% \usecolortheme[named=UBCblue]{structure}

% \makeatletter
% \setbeamertemplate{footline}
% {
%   \leavevmode%
%   \hbox{%
%   \begin{beamercolorbox}[wd=.4\paperwidth,ht=2.25ex,dp=1ex,center]{author in head/foot}%
%     \usebeamerfont{author in head/foot} \insertshortauthor %\hspace*{1em}(\insertshortinstitute)
%   \end{beamercolorbox}%
%   \begin{beamercolorbox}[wd=.5\paperwidth,ht=2.25ex,dp=1ex,center]{title in head/foot}%
%     \usebeamerfont{title in head/foot} \insertshorttitle
%   \end{beamercolorbox}%
%   \begin{beamercolorbox}[wd=.1\paperwidth,ht=2.25ex,dp=1ex,center]{date in head/foot}%
%     \usebeamerfont{date in head/foot}
%     \insertframenumber{} / \inserttotalframenumber\hspace*{2ex} 
%   \end{beamercolorbox}}%
%   \vskip0pt%
% }
% \makeatother

% \renewcommand{\arraystretch}{1.2}
% \renewcommand{\raggedright}{\leftskip=0pt \rightskip=0pt plus 0cm}
% \newcolumntype{C}[1]{>{\centering\let\newline\\\arraybackslash\hspace{0pt}}m{#1}}

% \hypersetup{
%     colorlinks=true,
%     linkcolor=UBCblue,
%     citecolor=UBCblue,
%     filecolor=magenta,      
%     urlcolor=blue,
%     allcolors=.
% }
% \setbeamercolor{button}{bg=UBCblue2,fg=white}
% \newcommand\fnote[1]{\captionsetup{font=tiny}\caption*{#1}}
% \newcommand\fnotev[1]{\captionsetup{font=scriptsize}\caption*{#1}}
% \setbeamertemplate{caption}[numbered]


% %\justifying
% \urlstyle{same}
%\usefonttheme{serif}

%------------------------
%------------------------

\date{}

%------------------------
%------------------------
%----------------------------------------------------------------------------------------
%	TITLE PAGE
%----------------------------------------------------------------------------------------------------------------
%------------------------

\title[Joe Fish Prospectus]{How Should We Think About Low Income Rental Markets} % The short title appears at the bottom of every slide, the full title is only on the title page
\author[Joe Fish]{Joe Fish}


\begin{document}

\begin{frame}
\titlepage % Print the title page as the first slide
\end{frame}

\begin{frame}{Dissertation Chapters}
    \begin{enumerate}
        \item The Economics of Low-Income Landlords (job market paper)
        \item Landlord Responses to Changes in Tenant Protections
        \item Are Large Investors A Source of Declining Rental Affordability: Evidence from Boston, MA and Houston, TX (field paper)
    \end{enumerate}
    
\end{frame}

\begin{frame}{Recap of Field Paper}
    
    
\end{frame}

\begin{frame}{Motivation}
    \begin{enumerate}
        \item About 80\% of low income renters both live in private market units \parencite{jchs_2024, nhpd2024profiles}
        \pause
        \item Low income landlords are both more punitive (lower threshold to evict) and more profitable \parencite{Desmond_2019, Eisfeldt_2015,Damen_2025}
        \begin{itemize}
            \item Suggests room to regulate the worst landlords
        \end{itemize}
        \pause
        \item At the same time, given reliance on private market and lack of funds for subsidized housing, optimal regulation has to be careful about inducing landlord exit
    \end{enumerate}

\end{frame}

\begin{frame}{Research questions}

\begin{enumerate}
    \item \textbf{What rationalizes high prices in low income rental markets?}
    \pause
    \item What do the exit decisions of low income landlords look like? Specifically, who are the \textbf{marginal} vs \textbf{inframarginal} landlords
    \pause
    \item Given knowledge of landlord markups and exit decisions, what's the best way to regulate the low income rental market?
\end{enumerate}

\end{frame}

\begin{frame}{This Presentation}
    \begin{enumerate}
        \item Present stylized facts about the Philadelphia Rental Market
        \begin{itemize}
            \item high prices consistent with prior literature
            \item evidence of market segmentation
        \end{itemize}
         \pause
        \item Very preliminary model to rationalize surprisingly high prices for low income units
        \pause
        \begin{itemize}
            \item Key intuition: Rental market looks more like the labor market than a generic product market $\rightarrow$ search frictions are going to be a large source of market power
        \end{itemize}
    \end{enumerate}
    
\end{frame}

\begin{frame}{Contributions}

\begin{itemize}
    \item One of the first datasets to have a comprehensive and representative unit-level rent panel of a major American rental market
    \item 
    \item First paper to provide landlord exit thresholds 
\end{itemize}
    
\end{frame}

\begin{frame}{Literature on Evictions and Low-Income Landlords}
    \textbf{Prices and Profits}
        \begin{itemize}
                \item Low Income landlords have \textbf{higher} profit margins than any other kind of landlord \parencite{Desmond_2019, Damen_2025, Eisfeldt_2015}
                \item Rent Prices are often as high as they are in much higher quality neighborhoods; maintenance is much lower 
        \end{itemize}
    \pause 
    \textbf{Landlord Business Models}
    \begin{itemize}
        \item ~9\% of rent goes unpaid (although it's unclear what happens to security deposits) \parencite{collinson2024eviction}
        \item Landlords allow about two months of back rent before filing to evict (\cite{collinson2024eviction} and Author's calculations)
        \begin{itemize}
            \item Intuition for why landlords wait: A large fraction of tenants default and then repay, so it's optimal for landlords to learn about tenant's probability of repayment
        \end{itemize}
    \end{itemize}
    
\end{frame}

\begin{frame}{Literature On Landlord Market Power}
TBD
    
\end{frame}


\section{Data}

\begin{frame}{Data and Institutional Setting}
\textbf{Philadelphia}
    \begin{itemize}
        \item Universe of Eviction Records from 2006-2019
        \begin{itemize}
            \item Contain \textbf{contract rents} as well case information (address, amount owed, plaintiff / defendant name, etc.)
            \item Cases are filtered to non-commercial and non-housing authority
        \end{itemize}
        \item Rental listing data from Altos (2011-2019)
        \item Rental Registry from 2016-2019 (covers about 90\% of rental units in Philadelphia)
    \end{itemize}
    \pause
    Why this is important:\\
    \begin{itemize}
        \item This will be one of the few papers to study the low income rental market because data are so sparse, otherwise 
    \end{itemize}
\end{frame}

\begin{frame}{Distribution of Rent Prices by Data Source}
    
\end{frame}




\section{Stylized Facts}


% \begin{frame}{Eviction in Low Income Rental Markets}
% \textbf{Eviction}
%     \begin{itemize}
%         \item Eviction is common: (~8-25 evictions / 100 renter households / year)
%         \item ~9\% of rent goes unpaid
%         \item Landlords allow about two months of back rent before filing to evict
%         \begin{itemize}
%             \item Intuition is a large fraction of tenants default and then repay; landlords learn about tenant's probability of repayment
%         \end{itemize}
%         \item Landlords anecdotally say they collect \$.10 for every \$1 in money judgment 
%         \item Most low income tenants rent from the private market
%         \begin{itemize}
%             \item ~20 million severely rent burdened households and about ~5 million subsidized housing units
%         \end{itemize}
%     \end{itemize}
%     \pause
    
% \end{frame}

\begin{frame}{Prices in Low Income Rental Markets}
\textbf{Hedonic Regressions}
    \begin{itemize}
    \item Intuitively, we should expect high filing rates (average number of evictions per unit per year) to be negatively correlated with rents
    \begin{itemize}
        \item High eviction rates may signal unobserved quality
        \item High eviction rates may signal lower threshold to evict
    \end{itemize}
    \pause
        \item Unconditionally, eviction rates don't correlate with prices after a ~20\% filing rate
        \pause
        \item After controlling for property and neighborhood characteristics, units with \textbf{higher} eviction rates charge \textbf{higher} rents 
    \end{itemize}
\end{frame}

\begin{frame}{Bin Scatter Eviction Rates vs Rents}
    
\end{frame}

\begin{frame}{Hedonic Regressions of Eviction Rates}
    \begin{center}
        \begin{longtable}{l|rrrr}
\caption*{
{\large Summary Statistics on Philadelphia Rent Prices} \\ 
{\small 2010-2019}
} \\ 
\toprule
\multicolumn{1}{l}{} & \multicolumn{2}{c}{Rent} & \multicolumn{2}{c}{Number of Evictions} \\ 
\cmidrule(lr){2-3} \cmidrule(lr){4-5}
\multicolumn{1}{l}{} & non-Top Evictor & Top Evictor & non-Top Evictor & Top Evictor \\ 
\midrule\addlinespace[2.5pt]
2010 & $675$ & $634$ & 14649 & 6674 \\ 
2011 & $685$ & $654$ & 14979 & 6498 \\ 
2012 & $700$ & $654$ & 15271 & 6642 \\ 
2013 & $700$ & $675$ & 15366 & 6064 \\ 
2014 & $725$ & $707$ & 15645 & 6257 \\ 
2015 & $750$ & $725$ & 14395 & 5891 \\ 
2016 & $750$ & $800$ & 14994 & 5622 \\ 
2017 & $775$ & $825$ & 14627 & 5748 \\ 
2018 & $800$ & $970$ & 11987 & 3735 \\ 
2019 & $850$ & $1,050$ & 11763 & 3473 \\ 
\bottomrule
\end{longtable}


    \end{center}
\end{frame}

\begin{frame}{Explanations for High Prices}
    \begin{itemize}
        \item High prices reflect high default risk
        \begin{itemize}
            \item Definitely true, but can't rationalize higher \textbf{profits} shown in related literature
        \end{itemize}
    \end{itemize}
    
\end{frame}

\begin{frame}{Market Segmentation in Low Income Rental Markets}
    
\end{frame}

% \begin{frame}{Share Regressions}
    
% \end{frame}


\begin{frame}{Rationalizing High Prices}
    \begin{itemize}
        \item Default Premia can explain higher prices but not higher profits
        \item Market Segmentation and lack of entry can explain higher prices but not price heterogeneity
        \item Market power can explain high prices but empirical estimates of changes in vacancies (quantity decreases) can't rationalize changes in prices
    \end{itemize}

\end{frame}

\begin{frame}{Models of Landlord Market power}

    
\end{frame}

\begin{frame}{Search Model}
    Intuition: In world where one landlord owned all properties in a market, 
\end{frame}









\end{document}
