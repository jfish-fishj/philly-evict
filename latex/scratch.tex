Research Questions:

what's the core econ question:
Do landlords need eviction?
Who responds to changes in tenant protections?
How do they respond? (changes in rent, who they rent to, exit market, etc.)
COVID?
What happens to tenants who end up in these properties



interesting facts
\begin{itemize}
    \item no discount on high evicting / shitty rentals
    \item how do landlords do/don't do maintenance?
    \item do thresholds for eviction vary by high / low evicting parcels
    
\end{itemize}

misc thoughts:
\begin{itemize}
    \item how far does allowing heterogenous costs of eviction get you in terms of rationalizing observed filing patterns?
\end{itemize}

model sketch
goal of model is to add power dynamics to a standard DDC model of landlord tenant relations. want to explain persistence of shitty, high price units. maybe why some units are top evictors?

the key mechanism is that tenants are dissuaded from making property complaints bc they know with some probability they will default in the future, in which case the landlord will be more punitive.

timing and model sketch
\begin{itemize}
    \item step 0: landlords match with tenants; sign contract for X months
    \item tenants and landlords draw $\epsilon$ and $\delta$ (default and something breaks) from CDFs
    \item based on draws, tenant does / doesn't default and thing does / doesn't break
    \item landlord decides whether to evict; tenant decides whether to complain
    \item if tenant decides to complain, next time tenant is in default, they are immediately evicted
    \begin{itemize}
        \item can add something where a certain percent of cases are bullshit, but we'll keep this simple for now
    \end{itemize}
\end{itemize}

some big thoughts on the project:

so right now, im struggling with whether the project seems "big enough" to do. a directed search model gets me what I want, to a certain extent, in that discriminating landlords get a premium over non-discriminating ones. it's kind of tricky for how to go from here.

the issues:

\begin{itemize}
    \item model not solved; have a shitty simulation working but doesn't quite give me what i want 
    \item need some sense of what vacancies/occupancy rates look like for low income landlords (surveys? talk to people? RHFS?)
    \item is philly enough? could expand to other cities with scraped craigslist data? does this cover the low income market? oakland rent registry? (issue w/ oakland would be getting eviction data...) berkeley rent registry? other ways to get low income rental data? Census?
    \item How to get market power back in the model?
\end{itemize}

TO-dos
\begin{itemize}
    \item get simulation working
    \item clean 311 data; clean complaint data. merge with philly
    \item scrape craigslist
\end{itemize}




\begin{table}[htbp]
   \caption{Price Change Regressions}
   \centering
   \begin{tabular}{lcccc}
      \tabularnewline \midrule \midrule
      Dependent Variable: & \multicolumn{4}{c}{Log Median Rent}\\
                                 & High Evictors & High Evictors (within Census Tract) & High Evictors (quintiles) & High Evictors (quintiles within Census Tract) \\   
      Model:                     & (1)           & (2)                                 & (3)                       & (4)\\  
      \midrule
      \emph{Variables}\\
      High Evictors              & 0.134$^{***}$ & 0.050$^{***}$                       &                           &   \\   
                                 & (0.014)       & (0.010)                             &                           &   \\   
      Filing Rate Quintile $=$ 2 &               &                                     & -0.015                    & 0.003\\   
                                 &               &                                     & (0.022)                   & (0.018)\\   
      Filing Rate Quintile $=$ 3 &               &                                     & 0.084$^{***}$             & 0.037$^{**}$\\   
                                 &               &                                     & (0.022)                   & (0.019)\\   
      Filing Rate Quintile $=$ 4 &               &                                     & 0.129$^{***}$             & 0.071$^{***}$\\   
                                 &               &                                     & (0.019)                   & (0.021)\\   
      Filing Rate Quintile $=$ 5 &               &                                     & 0.158$^{***}$             & 0.079$^{***}$\\   
                                 &               &                                     & (0.019)                   & (0.018)\\   
      \midrule
      \emph{Fixed-effects}\\
      parcel                     & Yes           & Yes                                 & Yes                       & Yes\\  
      year                       & Yes           &                                     & Yes                       & \\  
      Census Tract-year          &               & Yes                                 &                           & Yes\\  
      \midrule
      \emph{Fit statistics}\\
      Observations               & 51,901        & 51,901                              & 51,901                    & 51,901\\  
      \midrule \midrule
      \multicolumn{5}{l}{\emph{Clustered (parcel) standard-errors in parentheses}}\\
      \multicolumn{5}{l}{\emph{Signif. Codes: ***: 0.01, **: 0.05, *: 0.1}}\\
   \end{tabular}
\end{table}