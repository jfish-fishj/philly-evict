\documentclass[10pt, xcolor=dvipsnames]{beamer}

% There are many different themes available for Beamer. A comprehensive
% list with examples is given here:
% http://deic.uab.es/~iblanes/beamer_gallery/index_by_theme.html
% You can uncomment the themes below if you would like to use a different
% one:
%\usetheme{AnnArbor}
%\usetheme{Antibes}
%\usetheme{Bergen}
%\usetheme{Berkeley}
%\usetheme{Berlin}
%\usetheme{Boadilla}
%\usetheme{boxes}
%\usetheme{CambridgeUS}
%\usetheme{Copenhagen}
%\usetheme{Darmstadt}
%\usetheme{default}
%\usetheme{Frankfurt}
%\usetheme{Goettingen}
%\usetheme{Hannover}
%\usetheme{Ilmenau}
%\usetheme{JuanLesPins}
%\usetheme{Luebeck}
%\usetheme{Frankfurt}
\usetheme{metropolis}
%\usetheme{Malmoe}
%\usetheme{Marburg}
%\usetheme{Montpellier}
%\usetheme{PaloAlto}
%\usetheme{Pittsburgh}
%\usetheme{Rochester}
%\usetheme{Singapore}
%\usetheme{Szeged}
%\usetheme{Warsaw}

\usepackage[utf8]{inputenc}
\usepackage[english]{babel}
\usepackage{ragged2e}
\usepackage{bbding}
%\usepackage{enumitem}
\usepackage{mathtools}
\usepackage{indentfirst}
\usepackage{graphicx}
\usepackage{float}
\usepackage{hyperref}
\usepackage{mathtools}
\usepackage{preview}
\usepackage{xcolor}
\usepackage{color}
\usepackage{listings}
\usepackage{float}
\usepackage[caption = false]{subfig}
\usepackage{pdfpages}
\usepackage{multirow}
\usepackage{array}
\usepackage{makecell}
\usepackage{bm}
\usepackage{caption}
\usepackage{cancel}
\usepackage{anyfontsize}
\usepackage{etoolbox}
\usepackage{amsmath}
\usepackage{amssymb}
\usepackage{mathtools}
\usepackage{natbib}
\usepackage[flushleft]{threeparttable}
\usepackage{booktabs}
\usepackage{caption}
\usepackage{adjustbox}
\usepackage{appendixnumberbeamer}
\usepackage{pifont}
\usepackage{amsmath}
\usepackage{amssymb}
\usepackage[percent]{overpic}



\setbeamercolor{titlelike}{parent=structure}
\definecolor{UBCblue}{rgb}{0.04706, 0.13725, 0.32}
\colorlet{UBCblue2}{UBCblue!70!white}
\usecolortheme[named=UBCblue]{structure}

\makeatletter
\setbeamertemplate{footline}
{
  \leavevmode%
  \hbox{%
  \begin{beamercolorbox}[wd=.4\paperwidth,ht=2.25ex,dp=1ex,center]{author in head/foot}%
    \usebeamerfont{author in head/foot} \insertshortauthor %\hspace*{1em}(\insertshortinstitute)
  \end{beamercolorbox}%
  \begin{beamercolorbox}[wd=.5\paperwidth,ht=2.25ex,dp=1ex,center]{title in head/foot}%
    \usebeamerfont{title in head/foot} \insertshorttitle
  \end{beamercolorbox}%
  \begin{beamercolorbox}[wd=.1\paperwidth,ht=2.25ex,dp=1ex,center]{date in head/foot}%
    \usebeamerfont{date in head/foot}
    \insertframenumber{} / \inserttotalframenumber\hspace*{2ex} 
  \end{beamercolorbox}}%
  \vskip0pt%
}
\makeatother

\renewcommand{\arraystretch}{1.2}
\renewcommand{\raggedright}{\leftskip=0pt \rightskip=0pt plus 0cm}
\newcolumntype{C}[1]{>{\centering\let\newline\\\arraybackslash\hspace{0pt}}m{#1}}

\hypersetup{
    colorlinks=true,
    linkcolor=UBCblue,
    citecolor=UBCblue,
    filecolor=magenta,      
    urlcolor=blue,
    allcolors=.
}
\setbeamercolor{button}{bg=UBCblue2,fg=white}
\newcommand\fnote[1]{\captionsetup{font=tiny}\caption*{#1}}
\newcommand\fnotev[1]{\captionsetup{font=scriptsize}\caption*{#1}}
\setbeamertemplate{caption}[numbered]


%\justifying
\urlstyle{same}
%\usefonttheme{serif}

%------------------------
%------------------------

\date{}

%------------------------
%------------------------
%----------------------------------------------------------------------------------------
%	TITLE PAGE
%----------------------------------------------------------------------------------------------------------------
%------------------------

\title[Landlord Responses to Changes in Tenant Protections]{David Berger Meeting 2025-03-18} % The short title appears at the bottom of every slide, the full title is only on the title page
\author[Joe Fish]{Joe Fish}


\begin{document}

\begin{frame}
\titlepage % Print the title page as the first slide
\end{frame}

\begin{frame}{Meeting Agenda}
    \begin{itemize}
        \item Project Update / Change in Direction
        \item A couple empirical facts
        \item A sketch of a model
    \end{itemize}
    
\end{frame}

\begin{frame}{Project Update}
    \begin{itemize}
        \item Previous version focused on landlord market power. I might bring this back at some point to justify eviction protections, but it seemed convoluted and hard to discipline with data
        \item Current version is looking at how landlords respond to changes in tenant protections
        \item End goal is a  "how to do tenant protections correctly" paper
    \end{itemize}
    
\end{frame}

\begin{frame}{Institutional Setting}
    \begin{itemize}
        \item Philadelphia overhauled their eviction program in 2022
        \item I have evidence that this reduced evictions and raised prices for high evicting units (consistent with positive demand shock and/or cost pass through). 
        \begin{itemize}
            \item Ideally, I can pin price effects on a mechanism(s)
        \end{itemize}
    \end{itemize}
    
\end{frame}

\begin{frame}{Current Issues}
 Right now, I have a model that shows:
    \begin{itemize}
        \item Landlords exit when costs increase, with the degree of exit being constrained by outside option
        \item Price effects depend on how substitutable high / low quality rental units are
    \end{itemize}
The issue is that
\begin{itemize}
    \item Exit is a much more long term view of the rental market, whereas my price effects happen immediately. 
    \item The model doesn't really have a "reason" to be regulating eviction, beyond the fact that it's something cities are already doing
\end{itemize}
\end{frame}

\begin{frame}{Philly Eviction Changes in Three Slides}
    \begin{figure}
        \centering
        \includegraphics[width=0.75\linewidth]{figs/num_eviction_filings_by_year.png}
        \caption{Eviction Filings by Year}
        \label{fig:evict-year}
    \end{figure}
\end{frame}

\begin{frame}{Philly Eviction Changes in Three Slides}
    Here, I run an event study comparing prices of buildings (i) with a $>10\%$ filing rate per year. Coefficients are relative to 2019. I add census tract by year fixed effects to capture market-specific trends. Results are noisy, but indicate price effects
    \begin{align*}
        LogPrice_{it} = \sum_{t=2011}^{2023}\gamma_t*1[HighFiling_i] + \delta_i  +\theta_{nt}+\epsilon_{it}
    \end{align*}
    \small
    \begin{figure}
        \centering
        \includegraphics[width=0.65\linewidth]{figs/event_study_high_filing_preCOVID_m2.png}
        \caption{Event Study Results: Neighborhood Trends}
        \label{fig:placeholder}
    \end{figure}
    
\end{frame}

\begin{frame}{Evidence of Price Effects}
    \begin{figure}
        \centering
        \includegraphics[width=0.75\linewidth]{figs/mean_price_quintile.png}
        \caption{Price Changes over Time}
        \label{fig:price-high-filing}
    \end{figure}
    
\end{frame}



\begin{frame}{High Level Motivation of Model}
\begin{itemize}
    \item Goal of policymakers is to change landlord conduct (reduce evictions) while inducing minimal pass-through onto rents
    \item In the toy model, the key will be that certain landlords will be closer/farther away from exit/quality upgrading/deferred maintenance 
    \begin{itemize}
        \item Papers in the literature \cite{diamond-2019, collinson2024eviction, abramson_2021 } generally rely on an exit mechanism to move prices; I follow their lead
        \item To the extent price effects are because lower filing rates are an amenity, welfare depends on the price effect and the degree to which infrmarginal vs marginal tenants value this amenity
    \end{itemize}
    \begin{itemize}
        \item Exit ≈ quality upgrading (condo conversion), so policy effects propagate via reallocation across quality tiers
    \end{itemize}
\end{itemize}
    
\end{frame}

\begin{frame}{Toy Model Setup}
    \begin{itemize}
        \item Two period nested logit setup. Nests are low and high quality tiers
        \item At t=1, low tier landlords experience a cost shock and can pay a fee to enter the high tier or stay put and receive low tier flow profits
        \item Price effects will depend on covariance between cost shock and upgrade fee, plus degree of market segmentation
    \end{itemize}
    
\end{frame}

\begin{frame}{Toy Model}
    \begin{itemize}
        \item Two period model with discount rate $\delta =1$
        \pause
        \item Exogenous initial number of landlords (J) that each produce one unit of housing
        \pause
        \item Low nest ($L$) and high nest ($H$); each landlord is active in one nest
        \pause
        \item Market size $M$ is exogenous and normalized to $J=n_l +n_h$
        \pause
        \item At t=1, \textbf{low tier landlords} draw costs $\kappa_i>0$, and a quality upgrading cost $\phi_i>0$ from a joint CDF $F(\kappa,\phi)$ with correlation = $\rho$
        \pause
        \item \textbf{low tier landlords} can now pay an upgrade cost $\phi_i$ to enter the high tier (and avoid paying $\kappa_i$), or continue and make low tier flow profits
        \begin{itemize}
            \item Normalize marginal cost = 0, so $\pi_g=p_g$
        \end{itemize}
    \end{itemize}    
\end{frame}

\begin{frame}{Toy Model Cont.}
    Product $j$ in nest $g(j)$ has mean utility $\delta_j$; price coefficient $\alpha>0$; nest correlation (segmentation) $\sigma\in[0,1)$. \\
    
    Within-nest shares and inclusive values:
        \begin{align*}
            s_{j|g} \;=\;
            \frac{\exp\!\left(\frac{\delta_j-\alpha p_j}{1-\sigma}\right)}
            {\sum_{k\in g}\exp\!\left(\frac{\delta_k-\alpha p_k}{1-\sigma}\right)},
            \qquad
            I_g \;=\; (1-\sigma)\,\log\!\sum_{k\in g}\exp\!\left(\frac{\delta_k-\alpha p_k}{1-\sigma}\right).
        \end{align*}
    Nest shares (no outside):
        \begin{align*}
            S_g \;=\;\frac{e^{I_g}}{e^{I_L}+e^{I_H}},
            \qquad
            s_j \;=\; S_{g(j)}\,s_{j|g(j)},
            \qquad
            S_L+S_H \;=\; 1.
        \end{align*}
    
    Assuming products are homogenous within a nest: 
        \begin{align*}
            \boxed{\,s_j=\tfrac{1}{J},\;\; s_{j|g}=\tfrac{1}{n_g},\;\; S_g=\tfrac{n_g}{J}\,}.
        \end{align*}
    
\end{frame}

\begin{frame}{Nested Logit Pricing}
    With symmetry in nest $g$, the optimal \emph{per-product markup} is
        \begin{align*}
            p_g \;=\; \frac{s_j}{\alpha\big[1 - \sigma(1-s_{j|g}) - (1-\sigma)(1-S_g)\big]}.
        \end{align*}
plugging the symmetric shares $s_j=\tfrac{1}{J}$, $s_{j|g}=\tfrac{1}{n_g}$, $S_g=\tfrac{n_g}{J}$:
\begin{align*}
\boxed{
\,p_g 
\;=\;
\frac{1}{\alpha\,J\Big(1 - \sigma\!\big(1-\tfrac{1}{n_g}\big) - (1-\sigma)\!\big(1-\tfrac{n_g}{J}\big)\Big)}\;,\quad g\in\{L,H\}.
}
\end{align*}
Because $M=J$ and $s_j=1/J$, each seller’s quantity is $1$, so \emph{per-seller operating profit equals the markup}:
\begin{align*}
    \boxed{\,\Pi_g(n_L,n_H) \;=\; p_g \,}.
\end{align*}
    
\end{frame}

\begin{frame}{Nested Logit Comparative Statics}
    \textbf{Landlord choice (upgrade vs stay in L) with $(\kappa_i,\phi_i)$ correlated.}
    \pause
Given a draw $(\kappa_i,\phi_i)$:
\begin{align*}
\pi_i^{L} &= p_L - \kappa_i, 
&
\pi_i^{H} &= p_H-\phi_i.
\end{align*}

%For simplicity, assume $\pi_h = \pi_l$
Then the landlord upgrades iff $\kappa_i \geq \phi_i + (\pi_h - \pi_l)$\\
Define the set of landlords that upgrade as \[
\mathcal{U}(\Delta)\;\equiv\;\big\{\, i \in \mathcal{I}\;:\; \kappa_i \geq \phi_i+ (\pi_h - \pi_l) \,\big\}
\]
\pause
Intuitively, when $\rho$ is negative, you see more exit as higher cost shocks have better outside options. \\

Further, for a given amount of exit, the price effect is higher when the rental market is more segmented.

\end{frame}






\end{document}
