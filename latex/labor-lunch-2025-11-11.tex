\documentclass[10pt, xcolor=dvipsnames]{beamer}

\usetheme{metropolis}

\usepackage[utf8]{inputenc}
\usepackage[english]{babel}
\usepackage{ragged2e}
\usepackage{bbding}
\usepackage{mathtools}
\usepackage{indentfirst}
\usepackage{graphicx}
\usepackage{float}
\usepackage{hyperref}
\usepackage{preview}
\usepackage{xcolor}
\usepackage{amsmath, amssymb}
\usepackage{booktabs}
% If you are using biblatex in your main project, you can adapt this:
% \usepackage[style=authoryear]{biblatex}
% \addbibresource{references.bib}

% % Some helpful color shortcuts (feel free to tweak)
\definecolor{evictRed}{RGB}{180,30,50}
\definecolor{maintGreen}{RGB}{20,120,60}
\definecolor{phillyBlue}{RGB}{0,76,153}

\title[Landlord–Tenant Bargaining]{Landlord–Tenant Bargaining, Maintenance, and Retaliation}
\author[Joe Fish]{Joe Fish}
\date{} % leave blank for now

\begin{document}

%------------------------------------------------
\begin{frame}
  \titlepage
\end{frame}

------------------------------------------------
% \begin{frame}{Introduction}
%   \begin{itemize}
%     \item In this talk, I use administrative data and a stylized bargaining model to study how landlords and tenants interact in low income rental markets.
%     \pause
%     \item I will emphasize:
%       \begin{itemize}
%         \item How \textcolor{maintGreen}{maintenance bargaining} works in practice.
%         \item How \textcolor{evictRed}{retaliation via eviction filings} can shape that bargaining.
%       \end{itemize}
%     \pause
%     \item The goal is to connect \textbf{equilibrium facts} about maintenance and eviction to a simple \textbf{landlord–tenant bargaining model}.
%   \end{itemize}
%   % Suggestion: you might later tighten this to a 2–3 bullet “elevator pitch.”
% \end{frame}

%------------------------------------------------
\begin{frame}{Motivation}
  \textbf{*big picture*}
  \begin{itemize}
    \item there's been a recent focus on landlord market power on \textbf{*posted prices*} % <put some cites>
      \begin{itemize}
        \item these papers have a traditional quantity restriction mechanism, usually with landlords holding units vacant for longer
        \item landlord scale and concentration matter for rationalizing these effects
      \end{itemize}
    \pause
    \item however, many of the most important parts of a landlord-tenant relationship are \textbf{*bargained*} over
      \begin{itemize}
        \item maintenance requests, lease renewals, evictions
        \item large frictions, particularly in the low income segment, means that the structure of the bargaining process will be important
      \end{itemize}
  \end{itemize}
  % Suggestion: later you might standardize capitalization (“maintenance”, “ex ante”) and maybe break the first bullet into two sentences for readability.
\end{frame}

%------------------------------------------------
\begin{frame}{Research Question}
  \begin{center}
    \Large I'm going to try and convince you that bargaining is an important feature of the rental market
  \end{center}
\end{frame}

%------------------------------------------------
\begin{frame}{This Presentation}
  \begin{itemize}
    \item overview of landlord tenant law in Philadelphia as a framework for how this "bargaining" process works
    \pause
    \item stylized facts on equilibrium bargaining
      \begin{itemize}
        \item high evicting landlords do less maintenance, conditional on property characteristics \textbf{*and*} contract rents
        \item landlord-tenant bargaining breaks down
        \begin{itemize}
            \item landlords retaliate against tenants who make formal complaints about forgone maintenance
        \end{itemize}
      \end{itemize}
    \pause
    \item plea-bargaining style model to rationalize facts
  \end{itemize}
\end{frame}

%------------------------------------------------
\begin{frame}{Related Literature}
  \begin{itemize}
    \item plea bargains
    \item divorce law
    \item bargaining with asymmetric information
    \item quality regulation
    \item landlord tenant law
  \end{itemize}
\end{frame}

\section{Data and Context}

%------------------------------------------------
\begin{frame}{Overview of Low Income Rental Markets}
  \begin{itemize}
    \item tenant default is common; % <put that humphries cite about here>
      \begin{itemize}
        \item $\sim$95\% of cases involve non-payment of rent; median case is for \(\sim\)2 months back rent
        \item \(\sim 8\%\) of tenants have attorneys; \(\sim 80\%\) of landlords are
        \item outcome data is spotty, but in general tenants lose (\(\sim 95\%\) of cases)
      \end{itemize}
    \pause
    \item low income landlords generally have \textbf{*higher*} profit margins \parencite{Desmond_2019,Damen_2025}
      \begin{itemize}
        \item this is largely explained by lower costs, which will motivate my bargaining model
      \end{itemize}
    \item \textbf{*equilibrium has to rationalize higher profits and lower maintenance*}
  \end{itemize}
\end{frame}

%------------------------------------------------
\begin{frame}{Overview of Philadelphia Rental Market}
  \begin{itemize}
    \item Basic facts about the size and composition of the Philadelphia rental market.
    \item Distribution of landlord types (small vs large owners, geographic concentration).
    \item Heterogeneity in neighborhood characteristics and housing quality.
  \end{itemize}
  % Suggestion: later you can swap these placeholders for concrete numbers/figures.
\end{frame}

%------------------------------------------------
\begin{frame}{Overview of Philadelphia Rental Market (cont.)}
  \begin{itemize}
    \item Institutional details specific to Philadelphia:
      \begin{itemize}
        \item licensing and inspection regime,
        \item local eviction court process,
        \item complaint and enforcement mechanisms.
      \end{itemize}
    \pause
    \item These details will matter for interpreting the empirical results and for the modeling choices.
  \end{itemize}
\end{frame}

%------------------------------------------------
\begin{frame}{Data}
\textbf{Eviction Data}
    \begin{itemize}
        \item Universe of Eviction Records from 2006-2025
        \begin{itemize}
            \item Contain \textbf{contract rents} as well case information (address, amount owed, plaintiff / defendant name, etc.)
            \item Cases are filtered to non-commercial and non-housing authority
        \end{itemize}
    \end{itemize}
\pause
\textbf{Rental listing data from Altos (2011-2023)}
\begin{itemize}
    \item Unit level rental listings with information on amenities, beds, baths, etc.
\end{itemize}
\textbf{ Rental Registry}
\begin{itemize}
    \item Covers near-universe of rental listings
    \item Importantly, landlords need to be on the rental listing to file evictions
\end{itemize}
\textbf{Complaint Data}
\begin{itemize}
    \item 311 complaints made to the Licenses and Inspections Department in Philadelphia
    \item contains date of complaint, type of complaint (utilities shut off, vacant house, uncollected trash, etc), worker assigned to complaint, and eventual outcome (investigated; substantiated)
\end{itemize}
% Why this is important:\\
%     \begin{itemize}
%         \item This will be one of the few papers to study the low income rental market because data are so sparse 
%     \end{itemize}
\end{frame}

%------------------------------------------------

------------------------------------------------
\section{Stylized Facts}
\begin{frame}{Eviction in Philadelphia}
    \begin{itemize}
        \item Eviction is relatively infrequent
        \item A small share of properties have very high eviction rates
    \end{itemize}

    \begin{figure}
        \centering
        \includegraphics[width=0.75\linewidth]{figs/empirical_cdf_filing_rate_preCOVID.png}
        \caption{Empirical CDF of Filing Rate (pre-COVID)}
        \label{fig:ecdf-filings}
    \end{figure}
    
\end{frame}

%------------------------------------------------
\begin{frame}{Summary Statistics on Maintenance}
  \begin{table}[H]
    \centering
    \begin{tabular}{lccc}
      \toprule
      & Mean & SD & N \\
      \midrule
      % Fill in with actual variables
      Count of maintenance permits & & & \\
      Major structural repairs      & & & \\
      % ...
      \bottomrule
    \end{tabular}
    \caption{Summary Statistics on Maintenance}
  \end{table}
\end{frame}

%------------------------------------------------
\begin{frame}{Summary Statistics on Complaints}
  \begin{table}[H]
    \centering
    \begin{tabular}{lccc}
      \toprule
      & Mean & SD & N \\
      \midrule
      % Fill in with actual variables
      Complaint rate & & & \\
      Severe complaints share & & & \\
      % ...
      \bottomrule
    \end{tabular}
    \caption{Summary Statistics on Complaints}
  \end{table}
\end{frame}

%------------------------------------------------
\begin{frame}{Distribution of Rent Prices by Data Source}
Commercially available rental datasets (\textcolor{red}{altos}) don't cover the low income rental market\\
    \begin{figure}
        \centering
        \includegraphics[width=0.75\linewidth]{figs/density_rent_prices.png}
        \caption{Rent Prices by Data Source}
        \label{fig:rent-dist}
    \end{figure}
\end{frame}

\begin{frame}{Rent Price Hedonics}
  \begingroup
\centering
\begin{tabular}{lc}
   \tabularnewline \midrule \midrule
   Dependent Variable:     & Log Median Rent\\  
   Model:                  & (1)\\  
   \midrule
   \emph{Variables}\\
   Filing Rate: 10-20\%    & 0.0087\\   
                           & (0.0069)\\   
   Filing Rate: 20-30\%    & 0.0303$^{***}$\\   
                           & (0.0084)\\   
   Filing Rate: 30\%+      & 0.0277$^{**}$\\   
                           & (0.0108)\\   
   \midrule
   \emph{Fixed-effects}\\
   Census Block Group-year & Yes\\  
   Decade Built            & Yes\\  
   Number of Stories       & Yes\\  
   Data Source             & Yes\\  
   Quality Grade           & Yes\\  
   Exterior Condition      & Yes\\  
   Building Type           & Yes\\  
   \midrule
   \emph{Fit statistics}\\
   Observations            & 131,719\\  
   \midrule \midrule
   \multicolumn{2}{l}{\emph{Clustered (parcel) standard-errors in parentheses}}\\
   \multicolumn{2}{l}{\emph{Signif. Codes: ***: 0.01, **: 0.05, *: 0.1}}\\
\end{tabular}
\par\endgroup

\end{frame}

%------------------------------------------------
\begin{frame}{landlord maintenance}
  landlords do less maintenance, even conditional on rent prices 

  % \include{tab}
  \hyperlink{permits-other}{\beamergotobutton{Other permit types}}
  % Suggestion: you might later standardize capitalization and spelling (“maintenance”).
\end{frame}

\section{Complaints and Retaliation}
%------------------------------------------------
\begin{frame}{overview of complaint process}
  \begin{enumerate}
    \item tenant makes complaint to landlord
    \pause
    \item if landlord does not fix complaint, tenant can file a 311 complaint with the Licenses and Inspections Department
      \begin{itemize}
        \item while complaint has been lodged, tenant is allowed to withhold rent \textbf{*by putting it in an escrow account*}
        \item \textbf{*if tenant is current with rent*}, they cannot be retaliated against (e.g., the landlord cannot file an eviction against them)
      \end{itemize}
    \pause
    \item landlord has 35 days to fix issue, following inspection
    \item After 35 days, L+I should reinspect. If the property is still in violation, L+I should refer the case to the City Solicitor for court
    \item if violation is severe enough, tenant is evicted \textbf{*by the court*}
  \end{enumerate}
\end{frame}

%------------------------------------------------
\begin{frame}{landlord retaliation: specification}
  \textbf{LpDiD math about here}

  \medskip
  \textcolor{evictRed}{Local projection / event–study setup:}
  \begin{align}
    Y_{it}
      &= \sum_{k=-6}^{6} \beta_k \,\mathbf{1}\{q = k\}_{it}
       + \alpha_i + \lambda_t + \varepsilon_{it},
      \label{eq:lpdid}
  \end{align}
  where:
  \begin{itemize}
    \item \(Y_{it}\) is an indicator for whether a building filed an eviction in a year-quarter,
    \item \(\mathbf{1}\{q = k\}_{it}\) is an indicator for \(k\) quarters relative to the complaint,
    \item \(\alpha_i\) are building fixed effects, \(\lambda_t\) are time fixed effects.
  \end{itemize}

  \medskip
  assumption is that effects fade out after 6 quarters

  % Suggestion: you might later explicitly note the omitted baseline period and how you normalize \(\beta_k\).
\end{frame}

%------------------------------------------------
\begin{frame}{landlord retaliation results}
  Probability of landlord retaliation increases \(\sim 200\%\) following a complaint

  \medskip
  \begin{figure}[H]
    \centering
    \includegraphics[width=0.8\linewidth]{figs/retaliation_event_study.pdf}
    \caption{Event-study of Eviction Probability Around Complaints}
  \end{figure}

  \medskip
  \hyperlink{dlags}{\beamergotobutton{Distributed lag results}}
\end{frame}

%------------------------------------------------
\begin{frame}[label=dlags]{Distributed Lag Results}
  \begin{itemize}
    \item Estimate distributed lag model of complaints on eviction filings:
  \end{itemize}
  \begin{align}
    Y_{it}
      &= \sum_{\ell=0}^{6} \theta_\ell \,\text{Complaint}_{i,t-\ell}
       + \alpha_i + \lambda_t + \varepsilon_{it},
  \end{align}
  \begin{itemize}
    \item Focus on dynamic pattern of retaliation over time.
    \item Check robustness to alternative lag lengths and functional forms.
  \end{itemize}

  \medskip
  \begin{figure}[H]
    \centering
    \includegraphics[width=0.8\linewidth]{figs/retaliation_distributed_lags.pdf}
    \caption{Distributed Lag Effects of Complaints on Eviction Filings}
  \end{figure}
\end{frame}

%------------------------------------------------
\begin{frame}{landlord retaliation: grace period}
  Next, I repeat the same specification, but I look at the percentage of cases that are filed for one or fewer months backrent.

  \medskip
  Intuition: since a large fraction of tenants default, one margin landlords can move to legally retaliate is to give tenants less grace

  \medskip
  \begin{align}
    \text{PctShortBackRent}_{it}
      &= \sum_{k=-6}^{6} \gamma_k \,\mathbf{1}\{q = k\}_{it}
       + \alpha_i + \lambda_t + u_{it}.
  \end{align}

  \medskip
  \begin{figure}[H]
    \centering
    \includegraphics[width=0.8\linewidth]{figs/short_backrent_event_study.pdf}
    \caption{Share of Evictions with \(\leq 1\) Month Back Rent}
  \end{figure}
\end{frame}

%================================================
% Stylized Bargaining Model
%================================================

%------------------------------------------------
\begin{frame}{Environment: Landlord--Tenant Maintenance Bargaining}
  \begin{itemize}
    \item Parties: risk–neutral landlord \(L\) vs.\ risk–neutral tenant \(T\).
    \item Object of bargaining: a maintenance level \(m \ge 0\) (repairs, upkeep).
      \begin{itemize}
        \item Tenant likes maintenance; landlord dislikes it.
        \item Utilities are linear in \(m\).
      \end{itemize}
    \item Landlord makes a take–it–or–leave–it maintenance offer \(m\).
    \pause
    \item If bargaining fails, tenant files a complaint and
          landlord fully commits to retaliation (filing for eviction)
          whenever the tenant falls behind on rent.
    \item If an eviction case occurs, each side pays some cost:
      \begin{align*}
        c_L &> 0 &&\text{(landlord's cost in an eviction case)}, \\
        c_T &> 0 &&\text{(tenant's cost in an eviction case)}.
      \end{align*}
  \end{itemize}
\end{frame}

%------------------------------------------------
\begin{frame}{Information and Default Risk}
  \begin{itemize}
    \item Tenant privately knows the probability that they will default on rent:
      \begin{align*}
        p \in (a,b) \subset (0,1).
      \end{align*}
    \item Interpret \(p\) as the tenant's private information about the path of
          their future income.
    \item A tenant with belief \(p\) is a type-\(p\) tenant.
    \item Landlord does not observe \(p\); instead:
      \begin{align*}
        p \sim F(\cdot)
        \quad \text{with density} \quad f(\cdot),
      \end{align*}
      where \(f(p)>0\) on \((a,b)\) and \(f\) is continuous and differentiable.
    \item Assume landlord can fully commit to retaliating (filing for eviction)
          whenever the tenant falls behind on rent.
  \end{itemize}
\end{frame}

%------------------------------------------------
\begin{frame}{Timing of the Game}
  \begin{enumerate}
    \item Landlord chooses a maintenance offer \(m\) (a level of upkeep).
    \item Tenant observes \(m\) and (given type \(p\)) chooses:
      \begin{align*}
        &\text{Accept maintenance } m
        \quad \text{or} \quad
        \text{File a complaint (reject the offer).}
      \end{align*}
    \pause
    \item If the tenant files a complaint, the landlord commits to retaliation:
      \begin{itemize}
        \item If the tenant later defaults on rent, an eviction case occurs.
        \item If the tenant does not default on rent, the landlord must pay
              a transfer \(\tau\) to the tenant (e.g.\ remedial maintenance,
              concessions).
      \end{itemize}
  \end{enumerate}
\end{frame}

%------------------------------------------------
\begin{frame}{Payoffs When Bargaining Fails}
  After filing a complaint and landlord's commitment to retaliation:

  \begin{itemize}
    \item If the tenant defaults on rent (probability \(p\)):
      \begin{align*}
        \Pi_T^{\text{evict}}(p) &= -c_T, \\
        \Pi_L^{\text{evict}}(p) &= -c_L.
      \end{align*}
    \item If the tenant does \emph{not} default on rent (probability \(1-p\)):
      \begin{align*}
        \Pi_T^{\text{no default}} &= \tau, \\
        \Pi_L^{\text{no default}} &= -\tau.
      \end{align*}
  \end{itemize}

  \vspace{0.5em}
  So the tenant's expected payoff from filing a complaint (rejecting \(m\))
  given type \(p\) is
  \begin{align*}
    U_T^{\text{complain}}(p)
      &= (1-p)\tau + p(-c_T) \\
      &= (1-p)\tau - p c_T.
  \end{align*}
\end{frame}

%------------------------------------------------
\begin{frame}{Tenant's Acceptance Condition}
  \begin{itemize}
    \item Normalize utilities so that
      \begin{align*}
        U_T^{\text{accept}}(m) = m
      \end{align*}
      (tenant likes maintenance one-for-one).
    \item From the previous slide, if tenant files a complaint and landlord
          commits to retaliation, a type-\(p\) tenant gets
      \begin{align*}
        U_T^{\text{complain}}(p)
          &= (1-p)\tau - p c_T.
      \end{align*}
    \item Type-\(p\) tenant accepts maintenance \(m\) iff
      \begin{align*}
        m &\ge (1-p)\tau - p c_T.
      \end{align*}
  \end{itemize}
\end{frame}

%------------------------------------------------
\begin{frame}{Borderline Type and Acceptance Region}
  Define the \emph{borderline type} \(q(m)\) as the tenant type that is
  indifferent between accepting maintenance \(m\) and filing a complaint:
  \begin{align*}
    m
      &= (1-q(m))\tau - q(m)c_T.
  \end{align*}

  Solving for \(q(m)\):
  \begin{align*}
    m &= \tau - q(m)(\tau + c_T), \\
    q(m) &= \frac{\tau - m}{\tau + c_T}.
  \end{align*}

  \vspace{0.75em}
  Acceptance rule:
  \begin{align*}
    \text{Tenant of type }p\text{ accepts maintenance }m
      \quad\Longleftrightarrow\quad
      p \ge q(m).
  \end{align*}
\end{frame}

%------------------------------------------------
\begin{frame}{Probability of Maintenance Being Performed}
  For a given maintenance offer \(m\):
  \begin{itemize}
    \item Tenants with \(p \ge q(m)\) accept maintenance \(m\).
    \item Tenants with \(p < q(m)\) file a complaint and trigger the
          retaliation regime.
  \end{itemize}

  \vspace{0.5em}
  Thus:
  \begin{align*}
    \Pr(\text{maintenance performed} \mid m)
      &= \Pr\{p \ge q(m)\} \\
      &= 1 - F\big(q(m)\big), \\
    \Pr(\text{complaint and retaliation} \mid m)
      &= F\big(q(m)\big).
  \end{align*}
\end{frame}

%------------------------------------------------
\begin{frame}{Landlord's Payoffs}
  Let \(k(m)\) denote the landlord's cost of providing maintenance \(m\),
  with \(k'(m) > 0\) and \(k''(m) \ge 0\).

  \vspace{0.5em}
  \begin{itemize}
    \item If maintenance \(m\) is accepted:
      \begin{align*}
        \Pi_L^{\text{maint}}(m)
          &= -k(m).
      \end{align*}
    \item If tenant files a complaint and landlord commits to retaliation,
          then for a type-\(p\) tenant:
      \begin{align*}
        \Pi_L^{\text{complain}}(p)
          &= (1-p)(-\tau) + p(-c_L) \\
          &= -\big[(1-p)\tau + p c_L\big].
      \end{align*}
  \end{itemize}
\end{frame}

%------------------------------------------------
\begin{frame}{Landlord's Expected Payoff from Maintenance Offer \(m\)}
  Using the acceptance region \(p \ge q(m)\) and density \(f(\cdot)\):

  \begin{align*}
    A(m)
      &= \mathbb{E}[\Pi_L \mid \text{offer } m] \\
      &= \big[1 - F\big(q(m)\big)\big]\Pi_L^{\text{maint}}(m)
         + \int_{a}^{q(m)} \Pi_L^{\text{complain}}(p)\, f(p)\, dp.
  \end{align*}

  Substituting the payoff expressions:
  \begin{align*}
    A(m)
      &= \big[1 - F\big(q(m)\big)\big]\big(-k(m)\big)
         + \int_{a}^{q(m)} \big[-(1-p)\tau - p c_L\big] f(p)\, dp \\
      &= -\big[1 - F\big(q(m)\big)\big] k(m)
         - \int_{a}^{q(m)} \big[(1-p)\tau + p c_L\big] f(p)\, dp.
  \end{align*}

  \vspace{0.5em}
  The landlord chooses \(m\) to maximize \(A(m)\), trading off:
  \begin{itemize}
    \item Higher maintenance cost \(k(m)\) when tenants accept, vs.
    \item Greater probability of complaints, retaliation, and eviction costs
          when tenants reject.
  \end{itemize}
\end{frame}

%------------------------------------------------
\begin{frame}{Comparative Statics: Shifting Default Risk Up or Down}
  Consider a shift in the distribution of tenant default risk \(p\):
  \begin{align*}
    p &\sim F(\cdot) \quad\leadsto\quad p \sim G(\cdot),
  \end{align*}
  where \(G\) is an upward shift of \(F\) (all types more likely to default).

  \vspace{0.5em}
  Formally, for some \(\varepsilon > 0\),
  \begin{align*}
    G(x) &= F(x - \varepsilon),
  \end{align*}
  so the mean of \(p\) increases but higher moments are unchanged.

  \vspace{0.75em}
  In equilibrium:
  \begin{itemize}
    \item The landlord offers \textbf{lower maintenance}:
      \begin{align*}
        m^* &\downarrow,
      \end{align*}
      because the complaint/retaliation path is now worse for tenants.
    \item The \textbf{probability maintenance is performed} stays the same:
      \begin{align*}
        r^* &= 1 - F(q^*)
      \end{align*}
      is preserved by adjusting the cutoff \(q^*\).
  \end{itemize}
\end{frame}

% (You can add additional comparative statics slides here if desired.)

%================================================
% Recap
%================================================

% %------------------------------------------------
\begin{frame}{Recap}
  in equilibrium:
  \begin{itemize}
    \item low income landlords have higher profits
    \item high evicting landlords do less maintenance
    \item landlords retaliate against tenants that file complaints
  \end{itemize}

  \vspace{0.5em}
  \pause
  \textcolor{phillyBlue}{Next steps:}
  \begin{itemize}
    \item Map reduced-form estimates more tightly to model parameters.
    \item Explore counterfactual landlord responses to changes in Philadelphia tenant protections.
  \end{itemize}
\end{frame}

\section{Appendix}
%------------------------------------------------
\begin{frame}[label=permits-other]{Other Permit Types}
  \begin{itemize}
    \item Heterogeneity across different permit categories (e.g.\ electrical, plumbing, structural).
    \item Compare coefficients on high\_evicting for:
      \begin{itemize}
        \item health/safety-related permits,
        \item cosmetic/non-safety permits.
      \end{itemize}
  \end{itemize}
  \pause
  \begin{table}[H]
    \centering
    \begin{tabular}{lcc}
      \toprule
      Permit type & Coef. on high\_evicting & SE \\
      \midrule
      % Fill in with actual estimates
      Structural & & \\
      Electrical & & \\
      Plumbing   & & \\
      % ...
      \bottomrule
    \end{tabular}
    \caption{Maintenance by Permit Type}
  \end{table}
\end{frame}

\end{document}
